\chapter{Revisão Teórica}\label{teoria}

Este capítulo tem o objetivo de apresentar a base conceitual necessária para a
compreensão do trabalho e as abordagens adotadas por trabalhos relacionados
para os problemas envolvidos no desenvolvimento no sistema.

\section{Gerência de operações}

A área, ou função, de operações, é a parte de uma empresa dedicada à produção e
entrega de bens e serviços para seus clientes. Gerência de operações é a
atividade de administrar os recursos que executam essas tarefas~\cite[p.
4]{slack2010operations}. As próximas subseções buscam formalizar alguns dos
conceitos relacionados a gerência de operações relevantes para este trabalho.

\subsection{Operações como sistema de transformação}

A área de operações de uma empresa é a área responsável pelo processo de
produção da mesma, ou seja, o sistema de transformação responsável por entregar
valor aos clientes através de operações realizadas nas
matérias-primas~\cite{hubka2012theory}. Formalmente, um sistema de
transformação pode ser dividido em subsistemas como ilustrado na
Figura~\ref{diagramaTransf}.

Os sistemas humano e técnico, em conjunto também denominados sistema executor,
consistem nas pessoas e máquinas que, através da aplicação de energia,
materiais e conhecimento, guiam o operando através do processo de
transformação. Aos componentes do sistema executor será dado o nome de
\textit{recursos executores}, que serão o foco do que se deseja gerenciar neste
trabalho. As ações do sistema executor são guiadas pelo ambiente ativo do
sistema, composto pelo sistema de informação e o sistema de gerência e metas. O
sistema de gerência e metas define os parâmetros de controle para execução do
processo, como quais recursos executores devem ser utilizados e quando cada
etapa deve ser realizada, enquanto o sistema de informação é uma representação
do conhecimento operacional agregado ao sistema, ou o \textit{know-how}.

\begin{figure}[h]
	\centering
	\includegraphics[width=\textwidth]{transf}
	\caption{Elementos de um sistema de transformação.
        Adaptado de \citeonline{hubka2012theory}.\label{diagramaTransf}}
\end{figure}

A solução proposta, que atua no planejamento e controle dos recursos do
processo de produção está incluída, pela estrutura proposta em
\citeonline{hubka2012theory}, no sistema de gerência e metas e, portanto, os
esforços de automação devem se concentrar em implementar as interfaces
existentes entre este sistema e o processo.

Duas atividades, importantes e intimamente relacionadas de responsabilidade do
sistema de gerência e metas, são o planejamento e controle da produção, as
quais serão o foco deste trabalho.

\subsection{Atividades de planejamento e controle}

As atividades de planejamento e controle se preocupam em conciliar a demanda do
mercado pelos produtos da empresa com a capacidade da operação de produzi-los.
Essas atividades se diferenciam principalmente pelo tamanho do intervalo de
tempo entre a tomada da decisão e o evento afetado por
ela~\cite{slack2010operations}. Planejamento é uma formalização do que é
esperado que aconteça em algum ponto no futuro, enquanto o controle pode ser
considerado como o replanejamento dado que alguma das premissas do plano
original se mostrou falsa durante sua execução. Observado isso, essas
atividades serão, pela maior parte deste trabalho, simplesmente tratadas por
\textit{planejamento}.

Para guiar as decisões envolvidas no planejamento, são definidas metas
relacionadas a indicadores de performance da operação, cujas características
variam com o tempo. Planos em longo prazo utilizam indicadores agregados de
natureza tipicamente financeira, dado que a incerteza da demanda é grande
demais para realizar previsões detalhadas dos acontecimentos. Em curto e médio
prazo, quando a demanda já é conhecida ou mais facilmente prevista, são
utilizados indicadores mais granulares, tipicamente mais relacionados à
qualidade do serviço prestado pela operação. Indicadores de performance para
processos produtivos em geral se referem a 5 características da operação:
qualidade, custo, confiabilidade, flexibilidade e
velocidade~\cite{slack2010operations}.


\section{Mineração de processos de negócio}

A execução de um processo de negócio, explícito ou implícito, em ambiente
corporativo gera traços no formato de um log de eventos, seja por razões de
controle ou de auditoria. O objetivo da mineração de processos de negócio é
derivar automaticamente um modelo do processo a partir dos dados de eventos
gerados por um ou mais outros sistemas de informação
corporativos~\cite{van2016process}.  Este modelo pode, então, ser utilizado para
responder questões a respeito da estrutura, da performance e da conformidade do
processo a normas estabelecidas.  A abordagem de construção automática a partir
de dados tem o benefício de gerar um modelo que é, em geral, mais aderente à
realidade do que o gerado manualmente. Uma vez que o modelo formal é gerado,
podem ser criadas representações dele em diferentes níveis de abstração e
detalhe, para comunicação aos \textit{stakeholders}, sem perder essa
propriedade.

\begin{figure}[h]
	\centering
	\includegraphics[width=\textwidth]{processmining}
        \caption{Ilustração dos principais tipos de mineração de processos: descoberta, conformidade e melhoria. Adaptado de \citeonline{van2016process}.\label{mineracao}}
\end{figure}

No log de eventos, as diferentes instâncias do processo que surgem para atender
as demandas da corporação estão representadas por seus traços: séries de
eventos com uma identificação comum de instância. Parte importante do desafio
da mineração é mapear os traços de um mesmo processo a um modelo formal com
representação de estado que possa ser analisado, por exemplo uma cadeia de
Markov ou uma rede de Petri estendida.

A Figura~\ref{mineracao} ilustra os principais tipos de mineração de processos,
apontando diferentes motivações e técnicas comumente inclusas na área. Neste
modelo, os sistemas de software são conscientes dos processos com os quais
interagem no ambiente. Conforme o processo de negócio evolui no ambiente,
gerando novos registros de eventos, o modelo também se modifica, por sua vez
adaptando o funcionamento dos sistemas de software regidos por ele. O tipo mais
relevante para este trabalho é a descoberta, com objetivo de construir um
modelo para prever o comportamento do processo em condições simuladas.


\subsection{Modelos Preditivos de Performance de Processo}\label{teo_pred}

Baseado nas técnicas de mineração de processos, há um esforço na criação de
modelos preditores de desempenho que sejam precisos e adaptáveis o bastante
para uso no dia-a-dia da operação (vide, por exemplo~\citeonline{van2008cycle}
e \citeonline{folino2012discovering}). Uma vez criados, tais modelos poderiam ser
utilizados por exemplo na recomendação de configurações para execução das
atividades ou notificações de risco de quebra de SLAs.

Para definir formalmente o problema, denote-se alguns
elementos~\cite{folino2012discovering}:

\begin{itemize}
        \item Seja $T$ o universo fixo de todos os traços possíveis, representando instâncias completas ou não.
        \item Seja $L \subset T$ um log.
        \item Seja $E$ o universo de todos os eventos associados a traços $\tau \in T$ que podem estar contidos em um log.
        \item Seja $M \in \mathbb{R}^n$ o universo dos valores possíveis para as $n$ métricas de performance do processo.
\end{itemize}

O problema de definir um modelo de predição de performance é derivar do log $L$
uma função $\hat{\mu}: T \to M$ que associe a cada traço um valor para cada
métrica. Define-se ainda um traço $\tau \in T$ como uma tripla $\langle id,
\bar{a}, s \rangle$, em que $id$ é uma identificação única para a instância do
processo, $\bar{a}$ é o conjunto de dados associados à instância e $s$ é uma
sequência de eventos.

\begin{figure}[h]
	\centering
	\includegraphics[width=\textwidth]{contexto}
	\caption{Diferentes tipos de contexto do processo.
        Adaptado de \citeonline{van2016process}.\label{escopos}}
\end{figure}

Os dados em $\bar{a}$ representam o contexto no qual a instância é executada.
Este contexto pode incluir informações a respeito da demanda que causou o
início da instância, variáveis de controle, ou mesmo resultados de atividades
já concluídas. Além de terem funções e fontes diferentes, essas variáveis
também têm escopos diferentes. A Figura~\ref{escopos} ilustra essa diversidade
na natureza do contexto em que os eventos ocorrem e seu efeito na predição de
resultados.

Sendo $V_a, V_b, \ldots, V_n$ as variáveis às quais se tem acesso em relação ao
processo, cada uma com um domínio $D_a, D_b, \ldots, D_n$, tem-se que $\bar{a}
\in D_a \times D_b \times \cdots \times D_n$. Visto que $\hat{\mu}$, por
definição, está definido para instâncias incompletas, é necessário que sua
implementação computacional seja capaz de ser calculada na ausência de
informações ainda não determinadas no processo. As famílias de abordagens mais
utilizadas para estimar o valor de $m \in M$ em traços de instâncias
incompletas são~\cite{van2016process}:

\begin{enumerate}

    \item Associar a cada estado do processo, definido por um prefixo de um
            traço completo, um estimador do valor mais provável que $m$ assumirá
            quando a instância estiver completa.

    \item Estender o traço, criando eventos simulados de forma a completar a
            instância, calculando o indicador sobre o traço completo.

\end{enumerate}

A abordagem utilizada neste trabalho é da família 1, baseada em
\citeonline{van2016process}, onde o autor utiliza um modelo determinístico de
transições, cujos estados são anotados com previsões para as métricas
definidas. Neste trabalho, porém, dado que o objetivo é maximizar as métricas
através da manipulação de variáveis de controle, cada estado é anotado com um
modelo de regressão cujo domínio contém apenas as variáveis para as quais há
valores disponíveis no respectivo ponto do processo.
