\chapter{Estudo de caso}\label{resultados}

Obteve-se os dados operacionais dos processos de manutenção preventiva de uma
planta industrial no estado de São Paulo, cujo modelo segue o da
Figura~\ref{preventiva}. Os dados referem-se aos meses de junho, julho e agosto
de 2017, contendo informações a respeito de:

\begin{enumerate}

    \item Recursos que executaram as tarefas;

    \item Máquinas que foram objeto da manutenção;

    \item Natureza do trabalho (por exemplo: manutenção mecânica ou elétrica)

    \item Data e hora de início e fim da instância do processo.

    \item Esforço despendido nas tarefas.

\end{enumerate}

Este banco de dados não está no formato tradicional utilizado em mineração de
processos de negócio, por estar orientado não a eventos, mas a instâncias. Para
simplificar a análise dos resultados, escolheu-se um tipo de processo para
avaliar: a manutenção mecânica de fornos elétricos de uma das áreas da fábrica.
Um exemplo ilustrativo da estrutura do banco de dados estrutura encontra-se na
Figura~\ref{exemplobanco}.

\begin{figure}[h]
	\centering
	\includegraphics[width=\textwidth]{exemplo}
	\caption{Exemplo ilustrativo da estrutura do banco de dados após limpeza.\label{exemplobanco}}
\end{figure}

A coluna `manutencionista' contém um código de identificação dado a cada
recurso humano que atua na manutenção. A coluna `trabalho' é o esforço
despendido na tarefa de manutenção, medido em minutos. As colunas de horário de
início e fim se referem à duração total da instância, incluindo as tarefas de
aquisição de ferramentas necessárias e locomoção até o local.

\begin{figure}[h]
	\centering
	\includegraphics[width=\textwidth]{trabreal}
        \caption{Esforço médio despendido na tarefa de manutenção para cada manutencionista que a executou no período analisado.\label{trabreal}}
\end{figure}

Para avaliar o impacto que a aplicação do sistema de planejamento da alocação
de recursos executores teria neste processo, é preciso validar a hipótese de
que a escolha de recurso utilizado para executar as tarefas influencia na
performance do processo. Neste caso, a métrica utilizada será a quantidade de
trabalho despendido. A Figura~\ref{trabreal} mostra o esforço médio e seu
desvio padrão para cada manutencionista analisado, para a tarefa escolhida.
Como se pode verificar, não apenas a escolha de recurso implica, na média, em
uma diferença no esforço despendido, como também na consistência dos resultados
obtidos.

A seguir, criou-se o modelo do processo operacional. Dadas as restrições nos
dados disponíveis, o processo original foi simplificado em duas atividades,
como demonstrado na Figura~\ref{simplificado}. A tarefa de preparação foi
inclusa para poder considerar nas métricas a diferença entre a duração total da
instância do processo e o esforço despendido na tarefa de manutenção de fato.
Assim, a métrica definida associada a este processo foi a soma das durações das
duas tarefas, equivalente à duração total da instância.

\begin{figure}[h]
	\centering
	\includegraphics[width=\textwidth]{simplificado}
        \caption{Modelo simplificado do processo de manutenção preventiva para uso no sistema.\label{simplificado}}
\end{figure}

Treinando o modelo preditivo para a métrica de performance deste processo, com
uso de validação cruzada em \textit{3-fold}, obteve-se um coeficiente de
determinação $r^2~=~0.83$, considerável aceitável para o uso em planejamento.

\chapter{Conclusões}

O sistema proposto obteve resultados positivos na predição da performance
operacional do processo de manutenção preventiva mecânica dos fornos elétricos
na planta industrial estudada. O processo estudado, por ser notoriamente
simples, atendeu todas as hipóteses feitas a respeito de sua estrutura, e o
estudo dos dados validou para este processo a hipótese implícita de influência
das decisões de alocação de recursos executores no resultado operacional. O
modelo preditivo proposto obteve um resultado aceitável para, a elaboração de
planos de alocação. Porém, como não havia dados a respeito do planejamento das
atividades, não foi possível testar a qualidade de planos gerados com base
neste modelo.

Os resultados obtidos revelam oportunidades de estudos futuros. A citar:

\begin{itemize}

    \item Análise das características dos processos onde sistemas como este
            apresentam maior valor.

    \item Melhorias no algoritmo de planejamento para relaxar hipóteses
            estruturais feitas a respeito do processo operacional.

    \item Análise da performance do modelo preditivo de desempenho em processos
            mais complexos. Há evidências, como em
                \citeonline{folino2012discovering}, de que pode ser necessário
                um modelo mais complexo.

    \item Adoção de um modelo sensível a risco para o planejamento, como o
            proposto em \citeonline{freire2016extreme}

\end{itemize}
