\chapter{Aspectos Conceituais}
\section{Contextualização}
A área de operações de uma empresa é a área responsável pelo processo de produção da mesma, ou seja, o processo de transformação responsável por entregar valor aos clientes através de operações realizadas nas matérias-primas~\cite{hubka2012theory}. Os elementos que compõem um processo de transformação podem ser observados na Figura~\ref{diagramaTransf}. Indicadores de performance para processos produtivos em geral se referem a 5 características da operação: qualidade, custo, confiabilidade, flexibilidade e velocidade~\cite{slack2010operations}.

\begin{figure}[h]
	\centering
	\includegraphics[width=\textwidth]{transf}
	\caption{Elementos de um sistema de transformação~\cite{hubka2012theory}.\label{diagramaTransf}}
\end{figure}

Uma ferramenta que atue no planejamento e controle dos recursos do processo de produção está incluída, pela estrutura proposta em~\cite{hubka2012theory}, no sistema de gerência e metas, em oposição ao sistema de informação, onde se encontra o conhecimento operacional relacionado ao processo. Duas atividades importantes e intimamente relacionadas de responsabilidade do sistema de gerência e metas são o planejamento e controle da produção, as quais serão o foco deste trabalho.

\section{Revisão da Literatura}
Há muitos trabalhos que se ocupam do problema específico da alocação automática de recursos em processos, sem lidar com os aspectos corporativos e estratégicos da aplicação dessas técnicas. Os algoritmos existentes vão desde simples mecanismos de \textit{push} e \textit{pull}, que não levam questões de performance do processo em consideração~\cite{russell2004workflow}~\cite{dumas2005process}~\cite{russell2008work}~\cite{pesic2007modelling}, até esforços mais recentes na aplicação de MDPs a este problema, utilizando dados gerados na execução do processo em algoritmos de aprendizagem por reforço.~\cite{huang2011reinforcement}~\cite{liu2008semi}

Enquanto muitos outros algoritmos para alocação de recursos são estudados no contexto de sistemas computacionais auto-gerenciados, os derivados de MDPs têm apresentado performance consistentemente superior na gerência de processos de negócio, domínio onde há grande variedade na estrutura processual e de métricas.~\cite{zhang1995reinforcement}
