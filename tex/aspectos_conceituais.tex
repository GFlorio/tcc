\chapter{Aspectos Conceituais}
\section{Gerência de Operações}
A área de operações de uma empresa é a área responsável pelo processo de produção da mesma, ou seja, o processo de transformação responsável por entregar valor aos clientes através de operações realizadas nas matérias-primas~\cite{hubka2012theory}. Os elementos que compõem um processo de transformação podem ser observados na Figura~\ref{diagramaTransf}. Indicadores de performance para processos produtivos em geral se referem a 5 características da operação: qualidade, custo, confiabilidade, flexibilidade e velocidade~\cite{slack2010operations}.

\begin{figure}[h]
	\centering
	\includegraphics[width=\textwidth]{transf}
	\caption{Elementos de um sistema de transformação~\cite{hubka2012theory}.\label{diagramaTransf}}
\end{figure}

Uma ferramenta que atue no planejamento e controle dos recursos do processo de produção está incluída, pela estrutura proposta em~\cite{hubka2012theory}, no sistema de gerência e metas, em oposição ao sistema de informação, onde se encontra o conhecimento operacional relacionado ao processo. Duas atividades importantes e intimamente relacionadas de responsabilidade do sistema de gerência e metas são o planejamento e controle da produção, as quais serão o foco deste trabalho.

\subsection{Atividades de Planejamento e Controle}

\section{Problema de alocação de tarefas (TAP)}
\textbf{**TODO**}

\section{Mineração de processos}
\textbf{**TODO**}

\section{Planejamento Automático}
\textbf{**TODO**}
