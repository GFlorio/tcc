\section{Visão Engenharia}
Para atender os requisitos de segurança e disponibilidade exigidos nos módulos computacionais, será utilizada a infraestrutura apresentada na Figura~\ref{engenharia}.

\begin{figure}[h]
	\centering
	\includegraphics[width=\textwidth]{visaoengenharia}
	\caption{Diagrama de infraestrutura\label{engenharia}}
\end{figure}

Neste diagrama se observa a presença de um servidor interno à fábrica, necessário para garantir confiabilidade aos módulos Alocação e Comunicação, que caso contrário seriam vulneráveis a falhas na conexão da fábrica à internet.
O servidor externo agrega informações de performance das operações de diversas fábricas de forma a gerar relatórios executivos para utilização na camada estratégica.

Como o módulo estratégico está fora do escopo da primeira iteração, todos os componentes estarão contidos na rede local da fábrica e, portanto, sob a proteção de um firewall.

Decidiu-se utilizar um único servidor interno à fábrica porque, ainda que houvesse redundância no servidor de aplicação, o servidor banco de dados ainda seria um SPF\@. A duplicação em redundância ativa de toda a infraestrutura foi considerada muito custosa, mas poderá ser realizada caso haja demanda.
