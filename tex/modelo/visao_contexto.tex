\chapter{Desenvolvimento do Projeto}

\section{Contexto da aplicação}

Este sistema atua essencialmente na camada gerencial da operação, tornando o
projeto deste, até certo ponto, agnóstico do processo operacional que será
gerido. Para efeito de exemplificação, porém, será utilizado um processo de
referência para manutenção preventiva em uma planta industrial, demonstrado na
Figura~\ref{preventiva}. A escolha deste processo se deu por sua relativa
simplicidade e a fácil observação do valor que o sistema agrega neste contexto.

\begin{figure}[h]
	\centering
	\includegraphics[width=\textwidth]{preventiva}
	\caption{Processo de manutenção preventiva de fábrica.\label{preventiva}}
\end{figure}

Deste processo, considerando os objetivos operacionais discutidos
em \citeonline{slack2010operations}, pode-se derivar as seguintes métricas para
controle de performance:

\begin{itemize}
	\item Qualidade, como definido em \citeonline{wu2005preventive}.
	\item Custo, relacionado à quantidade de tempo que os recursos necessários para a tarefa ficam alocados para a sua realização.
	\item Lead Time, o tempo entre a identificação da necessidade de manutenção em um equipamento e o final desta manutenção.
\end{itemize}

O processo representado na Figura~\ref{preventiva} explicita o principal ator
que deve interagir com o sistema: O gerente da operação. Para este ator, que
realiza suas atividades de dentro da fábrica, o sistema auxilia na fase de
alocação de recursos executores às tarefas, fornecendo boas sugestões de planos
de alocação para atender às necessidades da operação.

A qualidade de um plano é dada com respeito a métricas anteriormente definidas
por um gestor de nível mais alto, responsável pelo planejamento estratégico de
médio e longo prazo da operação. Este ator interage com o sistema em um
primeiro momento na definição da estrutura do processo e das métricas
associadas a ele. Posteriormente, este gestor pode consultar indicadores
agregados baseados nos resultados da execução do processo para planejar
modificações na estrutura da operação ou em sua capacidade, dependendo das
previsões de negócio para a demanda.

Os resultados citados, em geral, são armazenados em algum tipo de sistema de
informação para apoio operacional, o que pode significar de uma planilha
eletrônica até um sistema completo de ERP\@. Esses dados precisam ser
conhecidos para que possam ser agregados em indicadores para o gestor
estratégico. Dado que esta tarefa tipicamente é realizada automaticamente
através de gatilhos de eventos no sistema fonte dos dados ou periodicamente por
um software especializado, pode-se incluir estes sistemas de informação entre
os atores.

Posto isso, pode-se dividir o software em duas partes em respeito ao uso: As
funcionalidades voltadas suporte ao planejamento de curto prazo da operação, e
as voltadas para a estruturação e planejamento de longo prazo. A
Figura~\ref{didatico} apresenta uma visão simplificada do uso típico do
sistema. A Figura~\ref{casosop} e a Figura~\ref{casosest} apresentam
os diagramas de caso de uso para os dois subsistemas descritos.

Observa-se pelos casos de uso que a necessidade de manter o modelo do processo
atualizado e sincronizado entre os subsistemas cria a necessidade de alguns
casos de uso adicionais, ativados periodicamente.

\begin{figure}[h]
	\centering
	\includegraphics[width=\textwidth]{didatico}
	\caption{Fluxo de informação no caso de uso principal.\label{didatico}}
\end{figure}

\begin{figure}[h]
	\centering
	\includegraphics[height=10cm]{casosop}
	\caption{Diagrama de casos de uso do subsistema operacional.\label{casosop}}
\end{figure}

\begin{figure}[h]
	\centering
	\includegraphics[height=10cm]{casosest}
	\caption{Diagrama de casos de uso do subsistema estratégico.\label{casosest}}
\end{figure}

