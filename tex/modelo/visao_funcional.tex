\section{Ponto de vista funcional}

A partir dos requisitos definidos anteriormente, pode-se derivar uma separação
do sistema em módulos, apresentados na Figura~\ref{pacotes} e definidos em detalhes a seguir. Os casos de uso que seguem o padrão \textit{CRUD} foram omitidos para preservar a clareza e relevância do modelo.

\paragraph{Arquitetura hexagonal} A estrutura interna de cada módulo segue o
padrão comumente denominado de \textit{arquitetura
hexagonal}~\cite{cockburn_hexagonal}~\cite{fowler2015layering}, uma derivação da
arquitetura clássica em três camadas que utiliza \textit{Data
Mappers}~\cite{fowler2002patterns} para eliminar dos objetos de domínio a
dependência da camada de acesso a dados.

\begin{figure}[t]
	\centering
	\includegraphics[]{pacotes}
	\caption{Diagrama de pacotes do sistema.\label{pacotes}}
\end{figure}

\begin{figure}[t]
	\centering
	\includegraphics[]{hexagonal}
        \caption{Diagrama esquemático da arquitetura hexagonal. Adaptado de \citeonline{fowler2015layering}.\label{hexagonal}}
\end{figure}

\subsection{Módulo Modelagem do Processo}

Este módulo é responsável por gerar o modelo computacional do processo
produtivo a partir da estrutura do processo e da informação sobre os instâncias passadas do processo. O modelo gerado será, então, utilizado como base para as outras funções do sistema, como o planejamento automático e a geração de relatórios.

\paragraph{Serviço gerar modelo operacional}
\begin{itemize}
	\item \textbf{Entradas:} Especificações do processo e dos recursos disponíveis.
	\item \textbf{Processamento:} Modelagem do processo de forma a tornar computacionalmente eficiente a simulação e análise.
	\item \textbf{Saída:} Modelo da operação.
\end{itemize}
\clearpage

\subsection{Módulo Alocação}

Este módulo é responsável por utilizar o modelo do processo operacional da
fábrica criado anteriormente para decidir a alocação ótima dos recursos para
atender a uma dada demanda, considerando a sua disponibilidade, características
específicas das tarefas habilitadas e a eficiência de longo prazo do processo.

Dado que o funcionamento adequado deste módulo é necessário para a continuidade da operação, garantir sua disponibilidade é um aspecto crítico do projeto.

\paragraph{Serviço planejar automaticamente}
\begin{itemize}
	\item \textbf{Entradas:} Demandas a que se deseja atender.
	\item \textbf{Processamento:} Determinar a melhor alocação dos recursos para atender às novas tarefas e atualizar o modelo interno.
	\item \textbf{Saída:} Plano otimizado
\end{itemize}


\subsection{Módulo Medição} 

Módulo responsável por recolher os dados
advindos da execução do processo e calcular as métricas de performance do
mesmo, de acordo com as especificações do negócio. Visto que os dados do
processo vêm de fontes muito heterogêneas, se justifica um módulo que abstrai
essa comunicação, oferecendo uma interface única para acesso às métricas.

As decisões gerenciais a respeito da alocação de recursos e composição de
portfólio serão tomadas a partir dos dados gerados por este módulo, portanto é
necessário garantir a segurança (safety) do mecanismo de coleta dos dados.

\paragraph{Serviço relatar atividade realizada}
\begin{itemize}
	\item \textbf{Entradas:} Dados dos sensores da operação.
	\item \textbf{Processamento:} Calcular métricas de qualidade e custo.
	\item \textbf{Saída:} Registro de relatório da atividade.
\end{itemize}

\paragraph{Serviço informar instâncias finalizadas}
\begin{itemize}
	\item \textbf{Entradas:} Processo a ser processado
	\item \textbf{Processamento:} Enviar para o subsistema estratégico as informações sobre novas instâncias terminadas do processo.
	\item \textbf{Saída:} Indicador de sucesso ou falha
\end{itemize}

\subsection{Módulo Análise de Impacto}

Módulo responsável por, utilizando o modelo do processo operacional, simular
sua performance em cenários arbitrários, de forma a auxiliar a gerência
operacional na estimativa de capacidade e otimização do portfólio de recursos,
a partir da simulação de cenários com mais ou menos recursos disponíveis.

\paragraph{Serviço calcula relatório}
\begin{itemize}
	\item \textbf{Entradas:} Intervalo de tempo para o qual calcular os relatórios
	\item \textbf{Processamento:} Calcular estatísticas de atraso de tarefas, utilização de recursos e eficiência, agregados no tempo.
	\item \textbf{Saída:} Relatório de desempenho da operação no cenário especificado.
\end{itemize}
