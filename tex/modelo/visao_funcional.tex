\section{Ponto de vista funcional}

A partir dos casos de uso definidos anteriormente, pode-se derivar uma
separação do sistema em módulos, apresentados na Figura~\ref{pacotes} e
definidos em detalhes a seguir. Os casos de uso que seguem o padrão
\textit{CRUD} foram omitidos para preservar a clareza e relevância do modelo.

\begin{figure}[h]
	\centering
	\includegraphics[]{pacotes}
	\caption{Diagrama de pacotes do sistema.\label{pacotes}}
\end{figure}

As principais funcionalidades do sistema estão contidas nos módulos
\textit{Modelagem do Processo}, no qual é definido o modelo preditivo de performance da operação e \textit{Alocação}, no qual é realizado o planejamento da alocação dos recursos.

\subsection{Módulo Modelagem do Processo}

Este módulo é responsável por gerar o modelo preditivo do processo
operacional a partir da estrutura do processo e da informação sobre os instâncias
passadas do processo. O modelo gerado será, então, utilizado como base para as
outras funções do sistema, como o planejamento automático e a geração de
relatórios.

\subsubsection{Serviço gerar modelo operacional}

\begin{itemize}

    \item \textbf{Pré-requisitos:} O processo, suas atividades e sua métrica de
            performance já devem ter sido cadastradas no sistema.

    \item \textbf{Entradas:} Processo para o qual se deseja gerar o modelo

    \item \textbf{Processamento:} Modelagem do processo de forma a tornar
            computacionalmente eficiente a simulação e análise.

	\item \textbf{Saída:} Modelo preditivo da operação.

\end{itemize}

A sequência de chamadas feita para execução deste serviço é apresentada na
Figura~\ref{seqmod}. A classe \textit{RepositorioExecucoes} contém a lógica de
recuperar do banco de dados as informações de instâncias já terminadas do
processo especificado. Esses dados recuperados consistem do contexto no qual as
tarefas foram executadas, como definido na Seção~\ref{teo_pred}, e os valores
obtidos para a métrica definida.

\begin{figure}[h]
	\centering
	\includegraphics[width=\textwidth]{seqmod}
    \caption{Diagrama de sequência da geração do modelo operacional.\label{seqmod}}
\end{figure}

Para o modelo de predição da performance da operação, foi utilizada a
implementação da biblioteca \textit{scikit-learn}~\cite{scikit-learn} do modelo
de floresta aleatória (\textit{random forest})~\cite{breiman2001random}. A
escolha por este modelo foi feita por causa de sua alta interpretabilidade por
humanos e capacidade de elencar a importância relativa das variáveis de entrada
para a predição, baseado nas posições das decisões que as envolvem em cada
árvore do conjunto. A Subseção~\ref{randomForest} detalha o funcionamento deste
modelo.

Para capturar comportamentos periódicos, quaisquer variáveis que representem
momentos no tempo são decompostas em:

\begin{itemize}
    \item Hora do dia;
    \item Dia da semana;
    \item Dia do mês;
    \item Mês do ano;
\end{itemize}

A informação do ano dos eventos é descartada, visto que, apesar de poder ajudar
na captura de tendências permanentes, seria preciso dados de muitos anos para
fazê-lo.

A especificação das métricas de performance operacional é feita através de
expressões numéricas onde se pode especificar nomes de variáveis que, quando a
métrica for calculada interpretada e calculada, serão substituídas pelo valor
da variável na instância pertinente. O interpretador de expressões implementado
utiliza notação polonesa. Uma exemplo de métrica possível é o custo do
trabalho para um processo de manutenção, que  pode ser especificada como:

\textit{\centering custoTrabalho = * manutencionista.custoHora + horasEsforco
horasLocomocao}

Neste exemplo, uma instância do processo que exigiu 1 hora de locomoção, 2
horas de esforço e foi executada por um recurso que custa R\$25,00 por hora
teria o valor $25.00 * (1 + 2) = 75.00$.

\subsubsection{Florestas Aleatórias}\label{randomForest}

O algoritmo utiliza um conjunto de árvores de decisão, cada uma treinada com
base em uma parte dos dados. Cada árvore é construída a partir de uma amostra
aleatória dos dados, e cada divisão é realizada em base em apenas um
subconjunto das variáveis de entrada (\textit{features}) disponíveis. Para
predição,  a saída do algoritmo é considerada como sendo a média das saídas das
árvores. Um exemplo de árvore de decisão gerada com os dados do estudo de caso
realizado neste trabalho está representada na Figura~\ref{arvore}.

\begin{figure}[h]
	\centering
	\includegraphics[width=\textwidth]{arvore}
        \caption{Grafo representando uma árvore de decisão gerada com os dados do estudo de caso.\label{arvore}}
\end{figure}

Cada nó não terminal da Figura~\ref{arvore} representa uma decisão a ser tomada
para determinação do valor final, enquanto os nós terminais representam valores
que a função pode assumir. As informações representadas em cada nó são:

\begin{enumerate}
    \item Opcionalmente, o teste a ser feito;
    \item O erro médio quadrático de representação das amostras que pertencem ao nó com o valor do nó;
    \item O número de amostras que pertencem ao nó;
    \item O valor numérico associado ao nó para a função alvo.
\end{enumerate}

\subsection{Módulo Alocação}

Este módulo é responsável por utilizar o modelo do processo operacional da
fábrica criado anteriormente para decidir a alocação ótima dos recursos
executores para atender a uma dada demanda, considerando a sua disponibilidade,
características específicas das tarefas habilitadas e a eficiência de longo
prazo do processo.

Dado que o funcionamento adequado deste módulo é necessário para a continuidade
da operação, garantir sua disponibilidade é um aspecto crítico do projeto.

\subsubsection{Serviço planejar automaticamente}

\begin{itemize}

    \item \textbf{Pré-requisitos:} O subsistema operacional já deve ter
            recebido do subsistema de gestão estratégica o modelo do processo
            para o qual se quer planejar.

	\item \textbf{Entradas:} Demandas a que se deseja atender.

    \item \textbf{Processamento:} Determinar a melhor alocação dos recursos
            para atender às novas tarefas e atualizar o modelo interno.

	\item \textbf{Saída:} Plano otimizado

\end{itemize}

\begin{figure}[h]
	\centering
	\includegraphics[width=\textwidth]{seqplan}
        \caption{Diagrama de sequência da elaboração automática do plano.\label{seqplan}}
\end{figure}

A Figura~\ref{seqplan} contém o diagrama de sequência da elaboração do plano
para atendimento das demandas. Partindo de uma lista de demandas a serem
atendidas, o algoritmo escolhe de forma gulosa (\textit{greedy}) executar a
atividade que traga a maior variação positiva da métrica pertinente. Se, ao
adicionar uma atividade ao plano, outras atividades forem habilitadas de acordo
com a lógica de controle de fluxo do processo, são inseridas novas demandas
para essas atividades na lista de demandas pendentes. Esta abordagem, cujo
pseudocódigo é apresentado no Algoritmo~\ref{pseudoplan} consegue evitar muitos
dos problemas de utilizar técnicas de planejamento clássico no problema de
alocação de recursos, como a necessidade de se representar explicitamente o
tempo das ações.~\cite{ghallab2016automated}

O algoritmo proposto, porém, preserva algumas das hipóteses do planejamento
clássico sobre a estrutura do processo:

\begin{enumerate}
    \item Determinismo das transições de estado.
    \item Ambiente finito e estático.
    \item Ambiente totalmente observável.
    \item Planejamento offline (o agente não conhece o estado de execução).
\end{enumerate}

\begin{algorithm}[h]
\caption{Determina a melhor próxima ação}\label{pseudoplan}
\begin{algorithmic}[1]
\Function{proximaAcao}{$instancias$, $inicio$}
    \State{$melhorAcao \gets NULL$}
    \While{$melhorAcao = NULL$}
        \State{$candidatas \gets \text{atividades habilitadas em } instancias$}
        \ForAll{$atividade \text{ in } candidatas$}
            \State{$impossivelExecutar \gets False$}
            \State{$papeisNecessarios \gets \text{papéis necessários para executar } atividade$}
            \ForAll{$papel \text{ in } papeisNecessarios$}
                \State{$papel.dominio \gets \text{recursos executores disponíveis para servir } papel \text{ em } inicio$}
                \If{$papel.dominio.size = 0$}
                    \State{$impossivelExecutar \gets True$}
                \EndIf{}
            \EndFor{}
            \If{$impossivelExecutar$}
            \State{\textbf{continue}}
            \EndIf{}
            \State{$metrica \gets atividade.processo.metricaPerformance()$}
            \State{$melhoresCondicoes \gets otimizador.otimiza(atividade.getVariaveis(), metrica)$}
            \State{$beneficioAcao \gets metrica(atividade, target)$}
        \EndFor{}
        \If{$melhorAcao = NULL$}
            \State{$inicio \gets inicio + timeStep$}
        \EndIf{}
    \EndWhile{}
\EndFunction{}
\end{algorithmic}
\end{algorithm}

Para determinar o impacto de adicionar uma atividade ao plano nas métricas, é
preciso determinar o melhor valor, que a atividade pode ter nas condições em
que é executada. Esse valor é estimado a partir do modelo preditivo do processo
e variando-se seus parâmetros de controle, como os recursos que executarão a
tarefa.

A otimização do valor de uma atividade é um problema de otimização discreta e
foi abordado utilizando busca local, mais precisamente escalada de gradiente
com reinícios aleatórios~\cite{russell1995modern}. Para efeito de determinação
dos passos possíveis no algoritmo, variáveis incontroláveis como o clima ou
características do pedido são consideradas como se tivessem apenas um elemento
em seus domínios: o valor desses parâmetros para a instância do processo em
questão.

\subsection{Módulo Medição} 

Módulo responsável por recolher os dados advindos da execução do processo e
calcular as métricas de performance do mesmo, de acordo com as especificações
do negócio. Visto que os dados do processo vêm de fontes muito heterogêneas, se
justifica um módulo que abstrai essa comunicação, oferecendo uma interface
única para acesso aos eventos.

As decisões gerenciais a respeito da alocação de recursos executores e composição de
portfólio serão tomadas a partir dos dados gerados por este módulo, portanto é
necessário garantir a segurança (safety) do mecanismo de coleta dos dados.

\subsubsection{Serviço relatar eventos}

\begin{itemize}

    \item \textbf{Pré-requisitos:} O subsistema operacional já deve ter
            recebido do subsistema de gestão estratégica o modelo do processo
            para o qual se quer planejar.

	\item \textbf{Entradas:} Dados de eventos ocorridos.

	\item \textbf{Processamento:} Agregar eventos em um estado global da operação.

	\item \textbf{Saída:} Registro de estado das atividades.

\end{itemize}

Este serviço é utilizado por agentes externos, como um processo periódico, para
relatar os novos eventos ocorridos na operação. Estes eventos são agregados em
um banco de dados de atividades planejadas, em execução e finalizadas. Quando
uma atividade final do processo é terminada, sua instância é movida para outo
banco de dados, para ser reportada ao subsistema de gestão estratégica.

\subsubsection{Serviço informar instâncias finalizadas}

\begin{itemize}

	\item \textbf{Entradas:} Processo a ser informado

    \item \textbf{Processamento:} Enviar para o subsistema de gestão
            estratégica as informações sobre novas instâncias terminadas do
            processo.

	\item \textbf{Saída:} Indicador de sucesso ou falha

\end{itemize}

Este serviço, também utilizado em geral por processos externos periódicos,
realiza a operação de transmitir as instâncias finalizadas para o subsistema de
gestão estratégica, através de uma chamada de rede.

\subsection{Módulo Análise de Impacto}

Módulo responsável por, utilizando o modelo do processo operacional, simular
sua performance em cenários arbitrários, de forma a auxiliar a gerência
operacional na estimativa de capacidade e otimização do portfólio de recursos executores,
a partir da simulação de cenários com mais ou menos recursos disponíveis.

\subsubsection{Serviço calcula relatório}

\begin{itemize}

    \item \textbf{Pré-requisitos:} Deve haver instâncias terminadas informadas
            pelo subsistema operacional.

    \item \textbf{Entradas:} Intervalo de tempo e processo para o qual calcular
            os relatórios
            
    \item \textbf{Processamento:} Calcular estatísticas de atraso de tarefas,
            utilização de recursos e eficiência, agregados no tempo.

    \item \textbf{Saída:} Relatório de desempenho da operação no cenário
            especificado.

\end{itemize}

O relatório, gerado com base no histórico de execução de tarefas e no modelo
operacional, é composto pelos seguintes indicadores:

\begin{enumerate}

    \item \textbf{Grau de utilização dos recursos:} A razão entre o tempo que
            cada recurso executor está alocado a uma tarefa e a soma da duração
            de seus turnos no período. Este indicador pode revelar um
            excesso de capacidade na operação.

    \item \textbf{Importância relativa das variáveis de controle:} O modelo
            preditivo do processo operacional é capaz de elencar a importância
            de cada variável na predição. Se as predições obtidas forem
            próximas da realidade, é esperado que essas variáveis tenham
            graus semelhantes de impacto na performance da operação. Este
            indicador pode guiar a elaboração de políticas de investimento
            em melhorias.

    \item \textbf{Estatísticas das métricas de desempenho especificadas:} Para
            a métrica definida no processo, é relevante saber sua variação ao
            longo do período indicado, seus valores máximo, médio e mínimo.
            Isso permite avaliar o cumprimento de metas operacionais.

\end{enumerate}

