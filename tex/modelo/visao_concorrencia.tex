\section{Ponto de vista da concorrência}

O principal ponto de atenção que deve ser observado com relação à concorrência
no projeto é a troca de dados entre os subsistemas operacional e gerencial. A
comunicação entre processos é feita por HTTP, sendo o subsistema operacional
que faz todas as requisições. Assim, o sistema exige apenas que esteja aberta a
porta 80 no protocolo TCP para conexões de saída no firewall da fábrica, uma
configuração que, por ser idêntica à utilizada para a web, é comumente
encontrada na prática.

Ainda assim, em caso de falha na transmissão dos dados de instâncias
completadas, estas devem continuar salvas no banco de dados para serem
reenviadas no próximo ciclo. Não é necessária uma política especial de
recuperação de falhas neste caso, desde que as todas as instâncias enviadas
sejam inseridas no banco de dados remoto como uma operação atômica.

Como existem potencialmente muitas implantações do subsistema operacional se
reportando a uma única instância do subsistema estratégico, é preciso que haja
algum mecanismo para evitar ou tratar colisões nos códigos de instâncias vindos
das fábricas. A falha em observar este ponto pode causar inconsistências no histórico de instâncias e, por consequência, a geração de modelos incorretos da operação, sobre os quais todas as outras funcionalidades do sistema se apoiam.
