\section{Ponto de vista do desenvolvimento}

O desenvolvimento será feito em Python. Com base nas bibliotecas
Django~\cite{django} para as funcionalidades baseadas em web e
Numpy~\cite{numpy} para cálculos numéricos. O estilo de código deve seguir o
definido pelo PEP 8~\cite{pep8}.

São obrigatórios testes funcionais a nível de serviço, ficando à discrição do
desenvolvedor os locais mais oportunos para utilizar também testes unitários.

\subsection{Arquitetura hexagonal} 

A estrutura interna de cada módulo segue o padrão comumente denominado de
\textit{arquitetura
hexagonal}~\cite{cockburn_hexagonal}~\cite{fowler2015layering}, uma derivação
da arquitetura clássica em três camadas que utiliza \textit{Data
Mappers}~\cite{fowler2002patterns} para eliminar dos objetos de domínio a
dependência da camada de acesso a dados. A estrutura geral deste padrão
arquitetural pode ser observada na Figura~\ref{hexagonal}.

\begin{figure}[h]
	\centering
	\includegraphics[]{hexagonal}
        \caption{Diagrama esquemático da arquitetura hexagonal. Adaptado de \citeonline{fowler2015layering}.\label{hexagonal}}
\end{figure}

