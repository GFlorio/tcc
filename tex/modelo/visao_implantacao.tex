\section{Ponto de vista da implantação}

Para atender os requisitos de segurança e disponibilidade exigidos nos módulos
computacionais, será utilizada a infraestrutura apresentada na
Figura~\ref{implantacao}.

\begin{figure}[h]
	\centering
	\includegraphics[]{implantacao}
	\caption{Diagrama de infraestrutura\label{implantacao}}
\end{figure}

Neste diagrama se observa a presença de um servidor interno à fábrica,
necessário para garantir confiabilidade aos módulos Alocação e Comunicação, que
caso contrário seriam vulneráveis a falhas na conexão da fábrica à internet.  O
servidor externo agrega informações de performance das operações de diversas
fábricas de forma a gerar relatórios executivos para utilização na camada
estratégica.

Decidiu-se utilizar um único servidor interno à fábrica porque, ainda que
houvesse redundância no servidor de aplicação, o servidor banco de dados ainda
seria um ponto único de falha. A duplicação em redundância ativa de toda a
infraestrutura foi considerada muito custosa, mas poderá ser realizada caso
haja demanda.

Os módulos da aplicação serão implantados na forma de contêineres Docker,
operando atrás de um servidor \textit{web} Nginx para tratar potenciais
clientes lentos e servir arquivos estáticos.
