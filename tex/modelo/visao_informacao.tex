\section{Ponto de vista da informação}

O modelo de referência para especificação de sistemas abertos de \textit{Big
Data} apresentado em \citeonline{chang2015nist} foi utilizado para determinar
as necessidades de armazenamento, processamento e transferência de dados neste
projeto. O resultado do mapeamento da arquitetura pode ser observado na
Figura~\ref{nist}.

\begin{figure}[h]
	\centering
	\includegraphics[width=\textwidth]{nist}
	\caption{Resultado do mapeamento da arquitetura do projeto ao modelo NIST.\label{nist}}
\end{figure}

O modelo permite observar as características de latência e fluxo de informações
necessários para o funcionamento adequado de cada módulo. São definidos dois
elementos necessários para prover os serviços de análise de dados: o provedor
de estrutura (\textit{Framework Provider}) e o provedor de aplicação
(\textit{Application Provider}).

As operações que cada módulo deve aplicar sobre os dados estão definidas na
especificação do provedor de aplicação, da onde se pode derivar características
da estrutura lógica dos dados para atender os requisitos funcionais do sistema.
A especificação do provedor de estrutura define \textit{quando} e \textit{como}
os dados são armazenados e acessados, o que auxilia na quantificação dos
aspectos não funcionais relacionados à infraestrutura.

\subsection{Estrutura lógica dos dados}

Os módulos do subsistema operacional possuem bancos de dados separados. O
subsistema de gestão estratégica possui, por razões descritas adiante, um banco
de dados unificado. A estrutura estrutura lógica de cada um está definida em
seus diagramas de Entidade-Relacionamento nas
figuras~\ref{ermod},~\ref{ermed}, e~\ref{eraloc}.

\subsection{Considerações de Escalabilidade}

Fisicamente, é previsto que o tamanho dos bancos de dados do subsistema de
Gestão Operacional cresça até um ponto de equilíbrio, dependente da
disponibilidade da conexão de rede com o subsistema de Gestão Estratégica, da
quantidade de tarefas no processo e na velocidade da operação. Atingido este
ponto, o espaço necessário para armazenar os dados operacionais deve se manter
constante.

O banco de dados unificado do subsistema de Gestão Estratégica tem uma
tendência constante de crescimento, causado pelo armazenamento de dados
históricos do processo operacional. Isso rege a necessidade de se determinar
uma política de retenção de dados durante a operação do sistema, a ser definida
de acordo com a disponibilidade de infraestrutura na implantação. Dado que é
previsto que o armazenamento deste banco de dados seja uma das maiores fontes
de custo operacional do sistema, optou-se por não duplicar os dados para os
dois módulos do subsistema Gestão Estratégica mantendo-os acoplados pela camada
de dados.

\begin{figure}[h]
	\centering
	\includegraphics[width=\textwidth]{ermod}
	\caption{Diagrama de Entidade-Relacionamento a nível de implementação do módulo Modelagem.\label{ermod}}
\end{figure}

\begin{figure}[h]
	\centering
	\includegraphics[width=\textwidth]{ermed}
	\caption{Diagrama de Entidade-Relacionamento a nível de implementação do módulo Medição.\label{ermed}}
\end{figure}

\begin{figure}[h]
	\centering
	\includegraphics[width=\textwidth]{eraloc}
	\caption{Diagrama de Entidade-Relacionamento a nível de implementação do subsistema de Gestão Estratégica.\label{eraloc}}
\end{figure}

