\chapter{Introdução}

\section{Motivação}
Segundo~\cite{mckinsey2017natural}, a adoção de novas tecnologias voltadas
ao aumento de eficiência na produção e consumo de recursos naturais tem potencial 
para causar uma economia de até 1.6 trilhões de dólares no cenário financeiro mundial 
até 2035.
Além disso uma parte relevante desses ganhos virá da otimização de processos nas
operações comerciais tanto dos produtores quanto dos consumidores destes recursos.
Neste cenário, evidencia-se a relevância da pesquisa e desenvolvimento em tecnologias
voltadas ao aumento de eficiência de operações.

Da mesma forma que a automação do trabalho operacional trouxe consistente aumento
na produtividade, há indícios de que isso possa ocorrer na camada gerencial das corporações.
Experimentos realizados em~\citeonline{gombolay2015decision} com times híbridos de humanos
e robôs, em ambiente controlado simulando uma linha de produção, verificaram o impacto
da introdução de elementos automáticos na distribuição e coordenação do trabalho.
Os resultados, apresentados na Figura~\ref{graficoMIT}, mostram que a automação da
distribuição do trabalho trouxe ganhos de eficiência operacional e, mais acentuadamente,
gerencial, resultante da tomada de decisões melhores e com menor custo.

O impacto econômico da adoção desse tipo de tecnologia pode ser estimado tendo em 
vista que, em média, 54 por cento do tempo produtivo de gerentes de todos os níveis
é gasto em atividades de coordenação e controle de recursos~\cite{accenture}, e 
que ferramentas de automação poderiam diminuir essa proporção para 25 por 
cento~\cite{accenture}, liberando aproximadamente 50 homens-hora por mês de cada gerente,
além de possíveis ganhos operacionais derivados da melhor qualidade das decisões.

Assim, um caminho viável para realizar parte dos ganhos econômicos previstos é através
de tecnologia de automação de decisão na gestão de operações, o que justifica o
desafio técnico de desenvolver tais tecnologias.

\begin{figure}[h]
	\centering
	\includegraphics[width=\textwidth]{resultadosMIT}
	\caption{Resultados da inserção de elementos automáticos na gerência~\cite{gombolay2015decision}.\label{graficoMIT}}
\end{figure}

\section{Objetivo}
O objetivo deste trabalho é elaborar um produto para auxiliar a tomada de decisões
relacionadas à alocação de recursos em processos operacionais industriais. O produto
baseará as decisões nos dados medidos do processo, de forma a prever a configuração
de recursos que otimiza as metas da operação.

Para atingir tais objetivos, o projeto de software se baseará nas técnicas apresentadas
em~\citeonline{rozanski2011software} para instanciar a arquitetura de referência 
para sistemas abertos de big data apresentada em \citeonline{chang2015nist}. O 
problema da otimização de processos será tratado utilizando-se técnicas de planejamento 
automático, como descrito em \citeonline{ghallab2016automated}.

\section{Revisão da Literatura}
Há muitos trabalhos que se ocupam do problema específico da alocação automática de
recursos em processos, sem lidar com os aspectos corporativos e estratégicos da
aplicação dessas técnicas. Os algoritmos existentes vão desde simples mecanismos
de \textit{push} e \textit{pull}, que não levam questões de performance do processo
em consideração~\cite{russell2004workflow, dumas2005process, russell2008work, pesic2007modelling},
até esforços mais recentes na aplicação de MDPs a este problema, utilizando dados
gerados na execução do processo em algoritmos de aprendizagem por reforço.~\cite{huang2011reinforcement, liu2008semi}

Enquanto muitos outros algoritmos para alocação de recursos são estudados no contexto
de sistemas computacionais auto-gerenciados, os derivados de MDPs têm apresentado
performance consistentemente superior na gerência de processos de negócio, domínio
onde há grande variedade na estrutura processual e de métricas.~\cite{zhang1995reinforcement}

\textbf{**DESENVOLVER MAIS**}

\section{Organização da Monografia}
\textbf{**TODO**}
