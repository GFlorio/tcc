\chapter{Introdução}
\section{Objetivo}
O objetivo deste trabalho é elaborar um produto para auxiliar a tomada de decisões relacionadas à alocação de recursos em processos operacionais industriais. O produto baseará as decisões nos dados medidos do processo, de forma a prever a melhor configuração de recursos que otimiza as metas da operação.

\section{Motivação}
A motivação acadêmica deste trabalho é estudar a aplicação de técnicas e métodos de Engenharia de Software no desenvolvimento de  projetos intensivos em processamento de dados.

Para a indústria, este trabalho tem o potencial de reduzir custos nas camadas operacionais e gerenciais, aumentando a eficiência do processo produtivo e, por consequência, a competitividade.

\section{Justificativa}
Este trabalho se justifica pelo potencial no aumento de produtividade nas operações industriais que a aplicação dos métodos pesquisados pode causar. Este potencial se comprova a partir de resultados obtidos em trabalhos anteriores e pesquisas de mercado.

Uma pesquisa feita pela Accenture\cite{accenture} com gerentes de diversos níveis e áreas de atuação revelou que, em média 54\% do tempo produtivo destes gerentes é gasto em atividades de coordenação e controle de recursos e que a adição de ferramentas de IA pode reduzir esse tempo para 25\%, liberando tempo desses profissionais para realizar outras tarefas.

O estudo apresentado em \cite{gombolay2015decision} realizou experimentos com times híbridos de humanos e robôs para avaliar o impacto da adição de elementos de automação gerencial na eficiência do time e na satisfação dos membros da equipe, chegando nos resultados apresentados na Figura \ref{graficoMIT}, onde se pode observar uma clara tendência de redução do tempo de montagem e de planejamento conforme a alocação de tarefas se torna mais autônoma.

\begin{figure}[h]
	\centering
	\includegraphics[width=\textwidth]{resultadosMIT}
	\caption{Resultados da inserção de elementos automáticos na gerência \cite{gombolay2015decision}.}
	\label{graficoMIT}
\end{figure}

Estes dados indicam, apesar de obtidos em situações de laboratório simulando linhas de produção, que há potencial para aumento de eficiência operacional e gerencial com a adição de automação na gerência de operações e, combinados com o resultados de \cite{accenture}, confirma-se o impacto econômico que este estudo pode trazer.