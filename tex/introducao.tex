\chapter{Introdução}

\section{Motivação}

Segundo~\cite{mckinsey2017natural}, a adoção de novas tecnologias voltadas ao
aumento de eficiência na produção e consumo de recursos naturais tem potencial
para causar uma economia de até 1.6 trilhões de dólares no cenário financeiro
mundial até 2035. Uma parte relevante desses ganhos virá da otimização de
processos nas operações industriais tanto dos produtores quanto dos
consumidores destes recursos. Neste cenário, evidencia-se a relevância da
pesquisa e desenvolvimento em tecnologias voltadas ao aumento de eficiência de
operações.

Da mesma forma que a automação do trabalho operacional trouxe consistente
aumento na produtividade, há indícios de que isso possa ocorrer na camada
gerencial das corporações.  Experimentos realizados em
\citeonline{gombolay2015decision} com times híbridos de humanos e robôs, em
ambiente acadêmico controlado simulando uma linha de produção, verificaram o
impacto da introdução de elementos automáticos na distribuição e coordenação do
trabalho.  Os resultados, apresentados na Figura~\ref{graficoMIT}, mostram que
a automação da distribuição do trabalho trouxe ganhos de eficiência operacional
e, mais acentuadamente, gerencial, resultante da tomada de decisões melhores e
em menor tempo.

O impacto econômico da adoção desse tipo de tecnologia pode ser estimado tendo
em vista que, em média, 54 por cento do tempo produtivo de gerentes de todos os
níveis é gasto em atividades de coordenação e controle de
recursos~\cite{accenture}, e que ferramentas de automação poderiam diminuir
essa proporção para 25 por cento~\cite{accenture}, liberando aproximadamente 50
homens-hora por mês de cada gerente, além de possíveis ganhos operacionais
derivados da melhor qualidade e maior consistência das decisões.

Assim, um caminho viável para realizar parte dos ganhos econômicos previstos é
através de tecnologia de automação de decisão na gestão de operações, o que
justifica o desafio técnico de desenvolver tais tecnologias.

\begin{figure}[h]
	\centering
	\includegraphics[width=\textwidth]{resultadosMIT}
	\caption{Resultados da inserção de elementos automáticos na gerência~\cite{gombolay2015decision}.\label{graficoMIT}}
\end{figure}

\section{Objetivos}

O objetivo deste trabalho é elaborar um sistema para auxiliar a tomada de
decisões relacionadas à alocação de recursos em processos operacionais,
especialmente industriais. O produto baseará as decisões nos dados medidos do
processo, de forma a prever a configuração de recursos que otimiza as metas da
operação.

As dificuldades que este tipo de sistema enfrenta estão relacionadas
principalmente com a natureza complexa e dinâmica dos processos de negócio
reais, cuja representação simplificada em um modelo pode deixar de considerar
aspectos relevantes. Para ser útil, é imprescindível que o sistema se baseie em
um modelo da operação que seja aderente à realidade.

Para atingir tais objetivos, o projeto de software se baseará nas técnicas
apresentadas em \citeonline{rozanski2011software} para instanciar a arquitetura
de referência para sistemas abertos de big data apresentada em
\citeonline{chang2015nist}. O problema da otimização de processos será tratado
utilizando-se técnicas de planejamento automático, como descrito em
\citeonline{ghallab2016automated}.

\section{Metodologia}

A metodologia do trabalho se baseia no modelo de três ciclos para pesquisa
em \textit{design science} descrito por \citeonline{hevner2007three} e
ilustrado na Figura~\ref{designScience}. Neste trabalho, os ciclos de
relevância e de projeto (\textit{design}, no artigo original) são executados
utilizando técnicas de especificação de arquitetura baseadas em pontos de vista
e perspectivas~\cite{rozanski2011software}.

\begin{figure}[h]
	\centering
	\includegraphics[width=\textwidth]{ciclos_design_science}
    \caption{Ciclos da pesquisa em \textit{design science}~\cite{hevner2007three}.\label{designScience}}
\end{figure}

\subsection{Correspondência entre os modelos}

Os sete pontos de vista propostos em \citeonline{rozanski2011software} e ilustrados na Figura~\ref{rozanski} são
associados aos elementos do modelo de \citeonline{hevner2007three} da seguinte
forma:

\begin{figure}[h]
	\centering
	\includegraphics[]{visoesrozanski}
        \caption{Pontos de vista utilizados na especificação da arquitetura. Adaptado de \citeonline{rozanski2011software}\label{rozanski}}
\end{figure}

\subsubsection{Ciclo de Relevância} O ponto de vista Contexto é a descrição das
hipóteses adotadas a respeito do domínio da aplicação de como o sistema
projetado se insere nesse contexto. A especificação deste ponto de vista inclui
características dos processos nos quais o sistema se insere, dos usuários e
outros sistemas técnicos com os quais ele deve interagir, e do impacto esperado
nesses processos como decorrência da implantação do sistema. Assim, como o
ciclo de relevância tem como objetivo justificar a construção dos artefatos do
projeto baseado exatamente no impacto que terão no domínio, a especificação do
ponto de vista Contexto foi considerada como o resultado da execução do ciclo
de relevância.

\subsubsection{Ciclo de Rigor} Dado que são objetivos do ciclo de rigor trazer da
academia informação a respeito do estado da arte para a construção dos
artefatos e critérios para avaliação dos mesmos, nota-se que essas atividades
foram realizadas na elaboração do embasamento teórico desta monografia. Assim,
os resultados da execução deste ciclo estão apresentados nos
capítulos~\ref{teoria} e~\ref{resultados} deste trabalho, apesar de não
encontrar correspondência nos pontos de vista previstos por
\citeonline{rozanski2011software}.

\subsubsection{Ciclo de Projeto} O ciclo de projeto, constituído da modelagem,
construção e avaliação dos artefatos documentais e funcionais do projeto de
software, resulta na especificação e implementação da arquitetura de software.
Estes aspectos estão presentes nos pontos de vista Funcional, Informação,
Concorrência, Desenvolvimento, Implantação e Operação, além das diferentes
perspectivas de qualidade consideradas neste trabalho. Destes pontos de vista,
Implantação e Operação estão presentes apenas como propostas para atender os
requisitos não-funcionais, não sendo efetivamente implementados.  


\section{Estrutura do trabalho}

Esta monografia está organizada em seis capítulos. O primeiro apresentou o
problema tratado, os objetivos que se espera atingir e a metodologia adotada
para tal.

O segundo capítulo faz uma revisão dos conceitos necessários para compreensão
do trabalho e apresenta abordagens adotadas em trabalhos relacionados.

No terceiro capítulo, é apresentado o contexto onde a aplicação se insere,
resultando na especificação de requisitos e, em seguida, o modelo
do sistema proposto para atender esses requisitos.

O quarto capítulo consiste na aplicação do software construído em um processo
simulado a partir de dados do processo de manutenção preventiva de uma grande
empresa brasileira da área de manufatura industrial de bens intermediários.

O quinto capítulo analisa os resultados obtidos no experimento realizado e
propõe melhorias para trabalhos subsequentes.
