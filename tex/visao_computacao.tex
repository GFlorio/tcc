\section{Visão Computação}
O domínio da aplicação pode ser traduzido em um diagrama de domínio~\cite{evans2004domain} como apresentado na Figura~\ref{dominio}.

\begin{figure}[t]
	\centering
	\includegraphics[width=\textwidth]{dominio}
	\caption{Diagrama de domínio da aplicação.\label{dominio}}
\end{figure}

A partir dos elementos do processo que se deseja automatizar, surge a definição dos módulos computacionais. A divisão desses módulos em camadas corporativas é demonstrado na Figura~\ref{corporativas}. O módulo de \textit{business intelligence} foi considerado fora do escopo para efeito do projeto de formatura.

\begin{figure}[t]
	\centering
	\includegraphics[width=\textwidth]{camadascorporativas}
	\caption{Divisão dos módulos do sistema em camadas corporativas.\label{corporativas}}
\end{figure}

A seguir, cada um desses módulos será detalhado, incluindo-se uma divisão em camadas como descrito em~\cite{evans2004domain}.

\subsection{Módulo Alocação}
Este módulo é responsável por utilizar o modelo do processo operacional da fábrica criado anteriormente para decidir a alocação ótima dos recursos, considerando a sua disponibilidade, características específicas das tarefas habilitadas e a eficiência de longo prazo do processo.

Dado que o funcionamento adequado deste módulo é necessário para a continuidade da operação, garantir sua disponibilidade é um aspecto crítico do projeto.

\begin{figure}[t]
	\centering
	\includegraphics[width=\textwidth]{camadas_alocacao}
	\caption{Arquitetura em camadas do módulo Alocação.\label{camadas_alocacao}}
\end{figure}

\paragraph{Serviço alocar recursos automaticamente}
\begin{itemize}
	\item \textbf{Entradas:} Novas tarefas habilitadas.
	\item \textbf{Processamento:} Determinar a melhor alocação dos recursos para atender às novas tarefas e atualizar o modelo interno.
	\item \textbf{Saída:} Mudanças a serem realizadas nos recursos.
\end{itemize}

\paragraph{Serviço alocar recursos manualmente}
\begin{itemize}
	\item \textbf{Entradas:} Tarefa habilitada, recursos a serem alocados.
	\item \textbf{Processamento:} Atualizar o modelo interno para atender às alocações fornecidas.
	\item \textbf{Saída:} Mudanças a serem efetivamente realizadas nos recursos.
\end{itemize}

\subsection{Módulo Medição}
Módulo responsável por recolher os dados advindos da execução do processo e calcular as métricas de performance do mesmo, de acordo com as especificações do negócio. Visto que os dados do processo vêm de fontes muito heterogêneas, se justifica um módulo que abstrai essa comunicação, oferecendo uma interface única para acesso às métricas.

As decisões gerenciais a respeito da alocação de recursos e composição de portfólio serão tomadas a partir dos dados gerados por este módulo, portanto é necessário garantir a segurança (safety) do mecanismo de coleta dos dados.

\begin{figure}[t]
	\centering
	\includegraphics[width=\textwidth]{camadas_medicao}
	\caption{Arquitetura em camadas do módulo Medição.\label{camadas_medicao}}
\end{figure}

\paragraph{Serviço relatar atividade realizada.}
\begin{itemize}
	\item \textbf{Entradas:} Dados dos sensores da operação.
	\item \textbf{Processamento:} Calcular métricas de qualidade e custo.
	\item \textbf{Saída:} Registro de relatório da atividade.
\end{itemize}

\paragraph{Serviço verificar estado atual do processo.}
\begin{itemize}
	\item \textbf{Entradas:} Processo a ser verificado
	\item \textbf{Processamento:} Compilar relatório de estado do processo.
	\item \textbf{Saída:} Relatório de estado atual do processo.
\end{itemize}

\paragraph{Serviço verificar métricas de desempenho.}
\begin{itemize}
	\item \textbf{Entradas:} Processo a ser verificado, período de tempo para cálculo.
	\item \textbf{Processamento:} Calcular as métricas de desempenho para o processo no período pedido.
	\item \textbf{Saída:} Relatório de métricas de desempenho.
\end{itemize}

\subsection{Módulo Comunicação}
Este módulo é responsável por instruir os agentes operacionais das decisões tomadas a respeito da alocação dos recursos. A diversidade de agentes envolvidos na operação, possivelmente exigindo integração do sistema central com diversos sistemas operacionais, justifica a existência de um módulo para abstrair detalhes dessas comunicações, oferecendo uma interface única para tal.

Dado que o funcionamento adequado deste módulo é necessário para a continuidade da operação, garantir sua disponibilidade é um aspecto crítico do projeto.

\begin{figure}[t]
	\centering
	\includegraphics[width=\textwidth]{camadas_comunicacao}
	\caption{Arquitetura em camadas do módulo Comunicação.\label{camadas_comunicacao}}
\end{figure}

\paragraph{Serviço comunicar alocação}
\begin{itemize}
	\item \textbf{Entradas:} Atividade a ser realizada, recursos que a realizarão.
	\item \textbf{Processamento:} Formatar os dados de forma a se integrar com as aplicações operacionais disponíveis aos agentes.
	\item \textbf{Saída:} Mensagem para as aplicações operacionais necessárias.
\end{itemize}

\subsection{Módulo Simulação}
Módulo responsável por, utilizando o modelo do processo operacional, simular sua performance em cenários arbitrários, de forma a auxiliar a gerência operacional na estimativa de capacidade e otimização do portfólio de recursos, a partir da simulação de cenários com mais ou menos recursos disponíveis.

\begin{figure}[t]
	\centering
	\includegraphics[width=\textwidth]{camadas_simulacao}
	\caption{Arquitetura em camadas do módulo Simulação.\label{camadas_simulacao}}
\end{figure}

\paragraph{Serviço simular cenário}
\begin{itemize}
	\item \textbf{Entradas:} Cenário a ser simulado
	\item \textbf{Processamento:} Criar uma cópia do modelo de produção com as alterações especificadas e executar uma simulação.
	\item \textbf{Saída:} Relatório de desempenho da operação no cenário especificado.
\end{itemize}

\paragraph{Serviço análise individual de recurso}
\begin{itemize}
	\item \textbf{Entradas:} Recurso a ser analisado
	\item \textbf{Processamento:} Compilar análise de impacto deste recurso na eficiência dos processos nos quais ele participa.
	\item \textbf{Saída:} Relatório de impacto do recurso nos processos.
\end{itemize}

\subsection{Módulo Gestão de Alocação}
Este módulo é responsável por gerar o modelo computacional do processo produtivo a partir da estrutura do processo e da informação sobre os recursos disponíveis.

\begin{figure}[t]
	\centering
	\includegraphics[width=\textwidth]{camadas_gestao_alocacao}
	\caption{Arquitetura em camadas do módulo Gestão da Alocação.\label{camadas_gestao_alocacao}}
\end{figure}

\paragraph{Serviço gerar modelo operacional}
\begin{itemize}
	\item \textbf{Entradas:} Especificações do processo e dos recursos disponíveis.
	\item \textbf{Processamento:} Modelagem do processo de forma a tornar computacionalmente eficiente a simulação e análise.
	\item \textbf{Saída:} Modelo da operação.
\end{itemize}

\subsection{Módulo adicional para validação}
Para efeitos de apresentação do sistema e realização dos testes de validação, será desenvolvido um módulo a mais, representando um sistema utilizado pelos agentes operacionais de um processo. Este módulo deverá ser capaz de receber notificações de alocação e registrar relatórios de atividades realizadas.

A figura~\ref{tela_operacional} mostra um protótipo da tela principal da aplicação.

\begin{figure}[t]
	\centering
	\includegraphics[height=300px]{tela_operacional}
	\caption{Protótipo da tela principal da aplicação de validação do sistema.\label{tela_operacional}}
\end{figure}
\clearpage
