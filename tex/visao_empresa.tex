\chapter{Especificação}
Para a especificação do sistema, será utilizada a técnica de pontos de vista do RM-ODP, que se relaciona com o processo de engenharia de software como demonstrado na Figura~\ref{rmODP}~\cite{Raymond1995}.

\begin{figure}[h]
	\centering
	\includegraphics[width=\textwidth]{rm-odp}
	\caption{Relação das visões do RM-ODP com o processo de software~\cite{Raymond1995}.\label{rmODP}}
\end{figure}

Deve-se notar que a visão tecnologia não corresponde à implementação de fato do sistema, mas sim à especificação de aspectos diretamente relacionados à implementação, como escolha de fornecedores de componentes e algoritmos específicos utilizados.

\section{Visão Empresa}
Este sistema atua essencialmente na camada gerencial da operação, tornando o projeto deste, até certo ponto, agnóstico do processo operacional que será gerido. Para efeito de teste e demonstração, porém, será utilizado um processo de referência demonstrado na Figura~\ref{preventiva}.
A escolha deste processo se deu por sua relativa simplicidade e a fácil observação do valor que o produto agrega neste contexto.

\begin{figure}[h]
	\centering
	\includegraphics[width=\textwidth]{preventiva}
	\caption{Processo de manutenção preventiva de fábrica.\label{preventiva}}
\end{figure}

Deste processo, considerando os objetivos operacionais discutidos em~\cite{slack2010operations}, pode-se derivar as seguintes métricas para controle de performance:
\begin{itemize}
	\item Qualidade, como definido em~\cite{wu2005preventive}.
	\item Custo, relacionado à quantidade de tempo que os recursos necessários para a tarefa ficam alocados para a sua realização.
	\item Lead Time, o tempo entre a identificação da necessidade de manutenção em um equipamento e o final desta manutenção.
\end{itemize}

O cálculo dessas métricas se dará a partir de dados coletados do uso dos recursos da operação, entre os quais: recursos humanos, ferramentas específicas e veículos.

