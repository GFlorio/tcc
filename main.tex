\documentclass[capsec]{politex}
% ========== Opções ==========
% pnumromarab - Numeração de páginas usando algarismos romanos na parte pré-textual e arábicos na parte textual
% abnttoc - Forçar paginação no sumário conforme ABNT (inclui "p." na frente das páginas)
% normalnum - Numeração contínua de figuras e tabelas 
%	(caso contrário, a numeração é reiniciada a cada capítulo)
% draftprint - Ajusta as margens para impressão de rascunhos
%	(reduz a margem interna)
% twosideprint - Ajusta as margens para impressão frente e verso
% capsec - Forçar letras maiúsculas no título das seções
% espacosimples - Documento usando espaçamento simples
% espacoduplo - Documento usando espaçamento duplo
%	(o padrão é usar espaçamento 1.5)
% times - Tenta usar a fonte Times New Roman para o corpo do texto
% noindentfirst - Não indenta o primeiro parágrafo dos capítulos/seções


% ========== Packages ==========
\usepackage[utf8]{inputenc}
\usepackage{amsmath,amsthm,amsfonts,amssymb}
\usepackage{graphicx,cite,enumerate}
\usepackage{helvet}
\usepackage[hidelinks]{hyperref}
\usepackage{algorithm}
\usepackage[noend]{algpseudocode}


\renewcommand{\familydefault}{\sfdefault}
\floatname{algorithm}{Algoritmo}
\graphicspath{{fig/}}

% ========== Language options ==========
\usepackage[brazil]{babel}
%\usepackage[english]{babel}


% ========== ABNT (requer ABNTeX 2) ==========
%	http://www.ctan.org/tex-archive/macros/latex/contrib/abntex2
\usepackage[alf]{abntex2cite}

% Forçar o abntex2 a usar [ ] nas referências ao invés de ( )
% \citebrackets[]

% ========== Opções do documento ==========
% Título
\titulo{Suporte à Decisão para Gerência de Operações}

% Autor
\autor{Gabriel Francisco Florio}

% Orientador / Coorientador
\orientador{Prof.\ Dr.\ Jorge Luis Risco Becerra}
% \coorientador{Nome do coorientador (opcional)}

% Tipo de documento
\tcc{de Computação}
%\dissertacao{Engenharia Elétrica}
%\teseDOC{Engenharia Elétrica}
%\teseLD
%\memorialLD

% Departamento e área de concentração
\departamento{PCS}
%\areaConcentracao{Engenharia de Computação}

\local{São Paulo}
\data{2017}




\begin{document}
% ========== Capa e folhas de rosto ==========
\capa{}
%\falsafolhaderosto{}
\folhaderosto{}


% ========== Folha de assinaturas (opcional) ==========
\begin{folhadeaprovacao}
	\assinatura{Prof.\ Dr.\ Jorge Luis Risco Becerra}
%	\assinatura{Prof.\ Y}
%	\assinatura{Prof.\ Z}
\end{folhadeaprovacao}


% ========== Ficha catalográfica ==========
% Fazer solicitação no site:
%	http://www.poli.usp.br/en/bibliotecas/servicos/catalogacao-na-publicacao.html


% ========== Dedicatória (opcional) ==========
%\dedicatoria{Dedicatória}


% ========== Agradecimentos ==========
%\begin{agradecimentos}

%Thanks...

%\end{agradecimentos}


% ========== Epígrafe (opcional) ==========
%\epigrafe{%
%	\emph{``Epígrafe''}
%	\begin{flushright}
%		-{}- Autor
%	\end{flushright}
%}


% ========== Resumo ==========
%\begin{resumo}
%Resumo...
%
%\\[3\baselineskip]
%
%\textbf{Palavras-Chave} -- Palavra, Palavra, Palavra, Palavra, Palavra.
%\end{resumo}


% ========== Abstract ==========
%\begin{abstract}
%Abstract...
%
%\\[3\baselineskip]
%
%\textbf{Keywords} -- Word, Word, Word, Word, Word.
%\end{abstract}


% ========== Listas (opcional) ==========
\listoffigures\relax
%\listadetabelas

% ========== Listas definidas pelo usuário (opcional) ==========
\begin{pretextualsection}{Lista de Abreviações e Siglas}
\begin{tabular}{ll}
	\textbf{CRUD} & Create, Retrieve, Update, Delete\\
	\textbf{ERP} & Enterprise Resource Planning\\
	\textbf{NIST} & National Institute of Standards and Technology\\
	\textbf{PEP} & Python Enhancement Proposal\\
	\textbf{SLA} & Service Level Agreement\\
\end{tabular}
\end{pretextualsection}
% ========== Sumário ==========
\sumario{}

% ========== Elementos textuais ==========
\chapter{Introdução}
\section{Objetivo}
O objetivo deste trabalho é elaborar um produto para auxiliar a tomada de decisões relacionadas à alocação de recursos em processos operacionais industriais. O produto baseará as decisões nos dados medidos do processo, de forma a prever a melhor configuração de recursos que otimiza as metas da operação.

\section{Motivação}
A motivação acadêmica deste trabalho é estudar a aplicação de técnicas e métodos de Engenharia de Software no desenvolvimento de  projetos intensivos em processamento de dados.

Para a indústria, este trabalho tem o potencial de reduzir custos nas camadas operacionais e gerenciais, aumentando a eficiência do processo produtivo e, por consequência, a competitividade.

\section{Justificativa}
Este trabalho se justifica pelo potencial no aumento de produtividade nas operações industriais que a aplicação dos métodos pesquisados pode causar. Este potencial se comprova a partir de resultados obtidos em trabalhos anteriores e pesquisas de mercado.

Uma pesquisa feita pela Accenture\cite{accenture} com gerentes de diversos níveis e áreas de atuação revelou que, em média 54\% do tempo produtivo destes gerentes é gasto em atividades de coordenação e controle de recursos e que a adição de ferramentas de IA pode reduzir esse tempo para 25\%, liberando tempo desses profissionais para realizar outras tarefas.

O estudo apresentado em \cite{gombolay2015decision} realizou experimentos com times híbridos de humanos e robôs para avaliar o impacto da adição de elementos de automação gerencial na eficiência do time e na satisfação dos membros da equipe, chegando nos resultados apresentados na Figura \ref{graficoMIT}, onde se pode observar uma clara tendência de redução do tempo de montagem e de planejamento conforme a alocação de tarefas se torna mais autônoma.

\begin{figure}[h]
	\centering
	\includegraphics[width=\textwidth]{resultadosMIT}
	\caption{Resultados da inserção de elementos automáticos na gerência \cite{gombolay2015decision}.}
	\label{graficoMIT}
\end{figure}

Estes dados indicam, apesar de obtidos em situações de laboratório simulando linhas de produção, que há potencial para aumento de eficiência operacional e gerencial com a adição de automação na gerência de operações e, combinados com o resultados de \cite{accenture}, confirma-se o impacto econômico que este estudo pode trazer.
\chapter{Revisão Teórica}\label{teoria}

Este capítulo tem o objetivo de apresentar a base conceitual necessária para a
compreensão do trabalho e as abordagens adotadas por trabalhos relacionados
para os problemas envolvidos no desenvolvimento no sistema.

\section{Gerência de operações}

A área, ou função, de operações, é a parte de uma empresa dedicada à produção e
entrega de bens e serviços para seus clientes. Gerência de operações é a
atividade de administrar os recursos que executam essas tarefas~\cite[p.
4]{slack2010operations}. As próximas subseções buscam formalizar alguns dos
conceitos relacionados a gerência de operações relevantes para este trabalho.

\subsection{Operações como sistema de transformação}

A área de operações de uma empresa é a área responsável pelo processo de
produção da mesma, ou seja, o sistema de transformação responsável por entregar
valor aos clientes através de operações realizadas nas
matérias-primas~\cite{hubka2012theory}. Formalmente, um sistema de
transformação pode ser dividido em subsistemas como ilustrado na
Figura~\ref{diagramaTransf}.

Os sistemas humano e técnico, em conjunto também denominados sistema executor,
consistem nas pessoas e máquinas que, através da aplicação de energia,
materiais e conhecimento, guiam o operando através do processo de
transformação. Aos componentes do sistema executor será dado o nome de
\textit{recursos executores}, que serão o foco do que se deseja gerenciar neste
trabalho. As ações do sistema executor são guiadas pelo ambiente ativo do
sistema, composto pelo sistema de informação e o sistema de gerência e metas. O
sistema de gerência e metas define os parâmetros de controle para execução do
processo, como quais recursos executores devem ser utilizados e quando cada
etapa deve ser realizada, enquanto o sistema de informação é uma representação
do conhecimento operacional agregado ao sistema, ou o \textit{know-how}.

\begin{figure}[h]
	\centering
	\includegraphics[width=\textwidth]{transf}
	\caption{Elementos de um sistema de transformação.
        Adaptado de \citeonline{hubka2012theory}.\label{diagramaTransf}}
\end{figure}

A solução proposta, que atua no planejamento e controle dos recursos do
processo de produção está incluída, pela estrutura proposta em
\citeonline{hubka2012theory}, no sistema de gerência e metas e, portanto, os
esforços de automação devem se concentrar em implementar as interfaces
existentes entre este sistema e o processo.

Duas atividades, importantes e intimamente relacionadas de responsabilidade do
sistema de gerência e metas, são o planejamento e controle da produção, as
quais serão o foco deste trabalho.

\subsection{Atividades de planejamento e controle}

As atividades de planejamento e controle se preocupam em conciliar a demanda do
mercado pelos produtos da empresa com a capacidade da operação de produzi-los.
Essas atividades se diferenciam principalmente pelo tamanho do intervalo de
tempo entre a tomada da decisão e o evento afetado por
ela~\cite{slack2010operations}. Planejamento é uma formalização do que é
esperado que aconteça em algum ponto no futuro, enquanto o controle pode ser
considerado como o replanejamento dado que alguma das premissas do plano
original se mostrou falsa durante sua execução. Observado isso, essas
atividades serão, pela maior parte deste trabalho, simplesmente tratadas por
\textit{planejamento}.

Para guiar as decisões envolvidas no planejamento, são definidas metas
relacionadas a indicadores de performance da operação, cujas características
variam com o tempo. Planos em longo prazo utilizam indicadores agregados de
natureza tipicamente financeira, dado que a incerteza da demanda é grande
demais para realizar previsões detalhadas dos acontecimentos. Em curto e médio
prazo, quando a demanda já é conhecida ou mais facilmente prevista, são
utilizados indicadores mais granulares, tipicamente mais relacionados à
qualidade do serviço prestado pela operação. Indicadores de performance para
processos produtivos em geral se referem a 5 características da operação:
qualidade, custo, confiabilidade, flexibilidade e
velocidade~\cite{slack2010operations}.


\section{Mineração de processos de negócio}

A execução de um processo de negócio, explícito ou implícito, em ambiente
corporativo gera traços no formato de um log de eventos, seja por razões de
controle ou de auditoria. O objetivo da mineração de processos de negócio é
derivar automaticamente um modelo do processo a partir dos dados de eventos
gerados por um ou mais outros sistemas de informação
corporativos~\cite{van2016process}.  Este modelo pode, então, ser utilizado para
responder questões a respeito da estrutura, da performance e da conformidade do
processo a normas estabelecidas.  A abordagem de construção automática a partir
de dados tem o benefício de gerar um modelo que é, em geral, mais aderente à
realidade do que o gerado manualmente. Uma vez que o modelo formal é gerado,
podem ser criadas representações dele em diferentes níveis de abstração e
detalhe, para comunicação aos \textit{stakeholders}, sem perder essa
propriedade.

\begin{figure}[h]
	\centering
	\includegraphics[width=\textwidth]{processmining}
        \caption{Ilustração dos principais tipos de mineração de processos: descoberta, conformidade e melhoria. Adaptado de \citeonline{van2016process}.\label{mineracao}}
\end{figure}

No log de eventos, as diferentes instâncias do processo que surgem para atender
as demandas da corporação estão representadas por seus traços: séries de
eventos com uma identificação comum de instância. Parte importante do desafio
da mineração é mapear os traços de um mesmo processo a um modelo formal com
representação de estado que possa ser analisado, por exemplo uma cadeia de
Markov ou uma rede de Petri estendida.

A Figura~\ref{mineracao} ilustra os principais tipos de mineração de processos,
apontando diferentes motivações e técnicas comumente inclusas na área. Neste
modelo, os sistemas de software são conscientes dos processos com os quais
interagem no ambiente. Conforme o processo de negócio evolui no ambiente,
gerando novos registros de eventos, o modelo também se modifica, por sua vez
adaptando o funcionamento dos sistemas de software regidos por ele. O tipo mais
relevante para este trabalho é a descoberta, com objetivo de construir um
modelo para prever o comportamento do processo em condições simuladas.


\subsection{Modelos Preditivos de Performance de Processo}\label{teo_pred}

Baseado nas técnicas de mineração de processos, há um esforço na criação de
modelos preditores de desempenho que sejam precisos e adaptáveis o bastante
para uso no dia-a-dia da operação (vide, por exemplo~\citeonline{van2008cycle}
e \citeonline{folino2012discovering}). Uma vez criados, tais modelos poderiam ser
utilizados por exemplo na recomendação de configurações para execução das
atividades ou notificações de risco de quebra de SLAs.

Para definir formalmente o problema, denote-se alguns
elementos~\cite{folino2012discovering}:

\begin{itemize}
        \item Seja $T$ o universo fixo de todos os traços possíveis, representando instâncias completas ou não.
        \item Seja $L \subset T$ um log.
        \item Seja $E$ o universo de todos os eventos associados a traços $\tau \in T$ que podem estar contidos em um log.
        \item Seja $M \in \mathbb{R}^n$ o universo dos valores possíveis para as $n$ métricas de performance do processo.
\end{itemize}

O problema de definir um modelo de predição de performance é derivar do log $L$
uma função $\hat{\mu}: T \to M$ que associe a cada traço um valor para cada
métrica. Define-se ainda um traço $\tau \in T$ como uma tripla $\langle id,
\bar{a}, s \rangle$, em que $id$ é uma identificação única para a instância do
processo, $\bar{a}$ é o conjunto de dados associados à instância e $s$ é uma
sequência de eventos.

\begin{figure}[h]
	\centering
	\includegraphics[width=\textwidth]{contexto}
	\caption{Diferentes tipos de contexto do processo.
        Adaptado de \citeonline{van2016process}.\label{escopos}}
\end{figure}

Os dados em $\bar{a}$ representam o contexto no qual a instância é executada.
Este contexto pode incluir informações a respeito da demanda que causou o
início da instância, variáveis de controle, ou mesmo resultados de atividades
já concluídas. Além de terem funções e fontes diferentes, essas variáveis
também têm escopos diferentes. A Figura~\ref{escopos} ilustra essa diversidade
na natureza do contexto em que os eventos ocorrem e seu efeito na predição de
resultados.

Sendo $V_a, V_b, \ldots, V_n$ as variáveis às quais se tem acesso em relação ao
processo, cada uma com um domínio $D_a, D_b, \ldots, D_n$, tem-se que $\bar{a}
\in D_a \times D_b \times \cdots \times D_n$. Visto que $\hat{\mu}$, por
definição, está definido para instâncias incompletas, é necessário que sua
implementação computacional seja capaz de ser calculada na ausência de
informações ainda não determinadas no processo. As famílias de abordagens mais
utilizadas para estimar o valor de $m \in M$ em traços de instâncias
incompletas são~\cite{van2016process}:

\begin{enumerate}

    \item Associar a cada estado do processo, definido por um prefixo de um
            traço completo, um estimador do valor mais provável que $m$ assumirá
            quando a instância estiver completa.

    \item Estender o traço, criando eventos simulados de forma a completar a
            instância, calculando o indicador sobre o traço completo.

\end{enumerate}

A abordagem utilizada neste trabalho é da família 1, baseada em
\citeonline{van2016process}, onde o autor utiliza um modelo determinístico de
transições, cujos estados são anotados com previsões para as métricas
definidas. Neste trabalho, porém, dado que o objetivo é maximizar as métricas
através da manipulação de variáveis de controle, cada estado é anotado com um
modelo de regressão cujo domínio contém apenas as variáveis para as quais há
valores disponíveis no respectivo ponto do processo.


\chapter{Desenvolvimento do Projeto}

\section{Contexto da aplicação}

Este sistema atua essencialmente na camada gerencial da operação, tornando o
projeto deste, até certo ponto, agnóstico do processo operacional que será
gerido. Para efeito de exemplificação, porém, será utilizado um processo de
referência para manutenção preventiva em uma planta industrial, demonstrado na
Figura~\ref{preventiva}. A escolha deste processo se deu por sua relativa
simplicidade e a fácil observação do valor que o sistema agrega neste contexto.

\begin{figure}[h]
	\centering
	\includegraphics[width=\textwidth]{preventiva}
	\caption{Processo de manutenção preventiva de fábrica.\label{preventiva}}
\end{figure}

Deste processo, considerando os objetivos operacionais discutidos
em \citeonline{slack2010operations}, pode-se derivar as seguintes métricas para
controle de performance:

\begin{itemize}
	\item Qualidade, como definido em \citeonline{wu2005preventive}.
	\item Custo, relacionado à quantidade de tempo que os recursos necessários para a tarefa ficam alocados para a sua realização.
	\item Lead Time, o tempo entre a identificação da necessidade de manutenção em um equipamento e o final desta manutenção.
\end{itemize}

O processo representado na Figura~\ref{preventiva} explicita o principal ator
que deve interagir com o sistema: O gerente da operação. Para este ator, que
realiza suas atividades de dentro da fábrica, o sistema auxilia na fase de
alocação de recursos executores às tarefas, fornecendo boas sugestões de planos
de alocação para atender às necessidades da operação.

A qualidade de um plano é dada com respeito a métricas anteriormente definidas
por um gestor de nível mais alto, responsável pelo planejamento estratégico de
médio e longo prazo da operação. Este ator interage com o sistema em um
primeiro momento na definição da estrutura do processo e das métricas
associadas a ele. Posteriormente, este gestor pode consultar indicadores
agregados baseados nos resultados da execução do processo para planejar
modificações na estrutura da operação ou em sua capacidade, dependendo das
previsões de negócio para a demanda.

Os resultados citados, em geral, são armazenados em algum tipo de sistema de
informação para apoio operacional, o que pode significar de uma planilha
eletrônica até um sistema completo de ERP\@. Esses dados precisam ser
conhecidos para que possam ser agregados em indicadores para o gestor
estratégico. Dado que esta tarefa tipicamente é realizada automaticamente
através de gatilhos de eventos no sistema fonte dos dados ou periodicamente por
um software especializado, pode-se incluir estes sistemas de informação entre
os atores.

Posto isso, pode-se dividir o software em duas partes em respeito ao uso: As
funcionalidades voltadas suporte ao planejamento de curto prazo da operação, e
as voltadas para a estruturação e planejamento de longo prazo. A
Figura~\ref{didatico} apresenta uma visão simplificada do uso típico do
sistema. A Figura~\ref{casosop} e a Figura~\ref{casosest} apresentam
os diagramas de caso de uso para os dois subsistemas descritos.

Observa-se pelos casos de uso que a necessidade de manter o modelo do processo
atualizado e sincronizado entre os subsistemas cria a necessidade de alguns
casos de uso adicionais, ativados periodicamente.

\begin{figure}[h]
	\centering
	\includegraphics[width=\textwidth]{didatico}
	\caption{Fluxo de informação no caso de uso principal.\label{didatico}}
\end{figure}

\begin{figure}[h]
	\centering
	\includegraphics[height=10cm]{casosop}
	\caption{Diagrama de casos de uso do subsistema operacional.\label{casosop}}
\end{figure}

\begin{figure}[h]
	\centering
	\includegraphics[height=10cm]{casosest}
	\caption{Diagrama de casos de uso do subsistema estratégico.\label{casosest}}
\end{figure}


\section{Ponto de vista funcional}

A partir dos requisitos definidos anteriormente, pode-se derivar uma separação
do sistema em módulos, apresentados na Figura~\ref{pacotes} e definidos em detalhes a seguir. Os casos de uso que seguem o padrão \textit{CRUD} foram omitidos para preservar a clareza e relevância do modelo.

\paragraph{Arquitetura hexagonal} A estrutura interna de cada módulo segue o
padrão comumente denominado de \textit{arquitetura
hexagonal}~\cite{cockburn_hexagonal}~\cite{fowler2015layering}, uma derivação da
arquitetura clássica em três camadas que utiliza \textit{Data
Mappers}~\cite{fowler2002patterns} para eliminar dos objetos de domínio a
dependência da camada de acesso a dados.

\begin{figure}[t]
	\centering
	\includegraphics[]{pacotes}
	\caption{Diagrama de pacotes do sistema.\label{pacotes}}
\end{figure}

\begin{figure}[t]
	\centering
	\includegraphics[]{hexagonal}
        \caption{Diagrama esquemático da arquitetura hexagonal. Adaptado de \citeonline{fowler2015layering}.\label{hexagonal}}
\end{figure}

\subsection{Módulo Modelagem do Processo}

Este módulo é responsável por gerar o modelo computacional do processo
produtivo a partir da estrutura do processo e da informação sobre os instâncias passadas do processo. O modelo gerado será, então, utilizado como base para as outras funções do sistema, como o planejamento automático e a geração de relatórios.

\paragraph{Serviço gerar modelo operacional}
\begin{itemize}
	\item \textbf{Entradas:} Especificações do processo e dos recursos disponíveis.
	\item \textbf{Processamento:} Modelagem do processo de forma a tornar computacionalmente eficiente a simulação e análise.
	\item \textbf{Saída:} Modelo da operação.
\end{itemize}
\clearpage

\subsection{Módulo Alocação}

Este módulo é responsável por utilizar o modelo do processo operacional da
fábrica criado anteriormente para decidir a alocação ótima dos recursos para
atender a uma dada demanda, considerando a sua disponibilidade, características
específicas das tarefas habilitadas e a eficiência de longo prazo do processo.

Dado que o funcionamento adequado deste módulo é necessário para a continuidade da operação, garantir sua disponibilidade é um aspecto crítico do projeto.

\paragraph{Serviço planejar automaticamente}
\begin{itemize}
	\item \textbf{Entradas:} Demandas a que se deseja atender.
	\item \textbf{Processamento:} Determinar a melhor alocação dos recursos para atender às novas tarefas e atualizar o modelo interno.
	\item \textbf{Saída:} Plano otimizado
\end{itemize}


\subsection{Módulo Medição} 

Módulo responsável por recolher os dados
advindos da execução do processo e calcular as métricas de performance do
mesmo, de acordo com as especificações do negócio. Visto que os dados do
processo vêm de fontes muito heterogêneas, se justifica um módulo que abstrai
essa comunicação, oferecendo uma interface única para acesso às métricas.

As decisões gerenciais a respeito da alocação de recursos e composição de
portfólio serão tomadas a partir dos dados gerados por este módulo, portanto é
necessário garantir a segurança (safety) do mecanismo de coleta dos dados.

\paragraph{Serviço relatar atividade realizada}
\begin{itemize}
	\item \textbf{Entradas:} Dados dos sensores da operação.
	\item \textbf{Processamento:} Calcular métricas de qualidade e custo.
	\item \textbf{Saída:} Registro de relatório da atividade.
\end{itemize}

\paragraph{Serviço informar instâncias finalizadas}
\begin{itemize}
	\item \textbf{Entradas:} Processo a ser processado
	\item \textbf{Processamento:} Enviar para o subsistema estratégico as informações sobre novas instâncias terminadas do processo.
	\item \textbf{Saída:} Indicador de sucesso ou falha
\end{itemize}

\subsection{Módulo Análise de Impacto}

Módulo responsável por, utilizando o modelo do processo operacional, simular
sua performance em cenários arbitrários, de forma a auxiliar a gerência
operacional na estimativa de capacidade e otimização do portfólio de recursos,
a partir da simulação de cenários com mais ou menos recursos disponíveis.

\paragraph{Serviço calcula relatório}
\begin{itemize}
	\item \textbf{Entradas:} Intervalo de tempo para o qual calcular os relatórios
	\item \textbf{Processamento:} Calcular estatísticas de atraso de tarefas, utilização de recursos e eficiência, agregados no tempo.
	\item \textbf{Saída:} Relatório de desempenho da operação no cenário especificado.
\end{itemize}

\section{Ponto de vista da informação}

O modelo de referência para especificação de sistemas abertos de \textit{Big
Data} apresentado em \citeonline{chang2015nist} foi utilizado para determinar
as necessidades de armazenamento, processamento e transferência de dados neste
projeto. O resultado do mapeamento da arquitetura pode ser observado na
Figura~\ref{nist}.

\begin{figure}[h]
	\centering
	\includegraphics[width=\textwidth]{nist}
	\caption{Resultado do mapeamento da arquitetura do projeto ao modelo NIST.\label{nist}}
\end{figure}

O modelo permite observar as características de latência e fluxo de informações
necessários para o funcionamento adequado de cada módulo. São definidos dois
elementos necessários para prover os serviços de análise de dados: o provedor
de estrutura (\textit{Framework Provider}) e o provedor de aplicação
(\textit{Application Provider}).

As operações que cada módulo deve aplicar sobre os dados estão definidas na
especificação do provedor de aplicação, da onde se pode derivar características
da estrutura lógica dos dados para atender os requisitos funcionais do sistema.
A especificação do provedor de estrutura define \textit{quando} e \textit{como}
os dados são armazenados e acessados, o que auxilia na quantificação dos
aspectos não funcionais relacionados à infraestrutura.

\subsection{Estrutura lógica dos dados}

Os módulos do subsistema operacional possuem bancos de dados separados. O
subsistema de gestão estratégica possui, por razões descritas adiante, um banco
de dados unificado. A estrutura estrutura lógica de cada um está definida em
seus diagramas de Entidade-Relacionamento nas
figuras~\ref{ermod},~\ref{ermed}, e~\ref{eraloc}.

\subsection{Considerações de Escalabilidade}

Fisicamente, é previsto que o tamanho dos bancos de dados do subsistema de
Gestão Operacional cresça até um ponto de equilíbrio, dependente da
disponibilidade da conexão de rede com o subsistema de Gestão Estratégica, da
quantidade de tarefas no processo e na velocidade da operação. Atingido este
ponto, o espaço necessário para armazenar os dados operacionais deve se manter
constante.

O banco de dados unificado do subsistema de Gestão Estratégica tem uma
tendência constante de crescimento, causado pelo armazenamento de dados
históricos do processo operacional. Isso rege a necessidade de se determinar
uma política de retenção de dados durante a operação do sistema, a ser definida
de acordo com a disponibilidade de infraestrutura na implantação. Dado que é
previsto que o armazenamento deste banco de dados seja uma das maiores fontes
de custo operacional do sistema, optou-se por não duplicar os dados para os
dois módulos do subsistema Gestão Estratégica mantendo-os acoplados pela camada
de dados.

\begin{figure}[h]
	\centering
	\includegraphics[width=\textwidth]{ermod}
	\caption{Diagrama de Entidade-Relacionamento a nível de implementação do módulo Modelagem.\label{ermod}}
\end{figure}

\begin{figure}[h]
	\centering
	\includegraphics[width=\textwidth]{ermed}
	\caption{Diagrama de Entidade-Relacionamento a nível de implementação do módulo Medição.\label{ermed}}
\end{figure}

\begin{figure}[h]
	\centering
	\includegraphics[width=\textwidth]{eraloc}
	\caption{Diagrama de Entidade-Relacionamento a nível de implementação do subsistema de Gestão Estratégica.\label{eraloc}}
\end{figure}


\section{Ponto de vista da concorrência}

O principal ponto de atenção que deve ser observado com relação à concorrência
no projeto é a troca de dados entre os subsistemas operacional e gerencial. A
comunicação entre processos é feita por HTTP, sendo o subsistema operacional
que faz todas as requisições. Assim, o sistema exige apenas que esteja aberta a
porta 80 no protocolo TCP para conexões de saída no firewall da fábrica, uma
configuração que, por ser idêntica à utilizada para a web, é comumente
encontrada na prática.

Ainda assim, em caso de falha na transmissão dos dados de instâncias
completadas, estas devem continuar salvas no banco de dados para serem
reenviadas no próximo ciclo. Não é necessária uma política especial de
recuperação de falhas neste caso, desde que as todas as instâncias enviadas
sejam inseridas no banco de dados remoto como uma operação atômica.

Como existem potencialmente muitas implantações do subsistema operacional se
reportando a uma única instância do subsistema estratégico, é preciso que haja
algum mecanismo para evitar ou tratar colisões nos códigos de instâncias vindos
das fábricas. A falha em observar este ponto pode causar inconsistências no histórico de instâncias e, por consequência, a geração de modelos incorretos da operação, sobre os quais todas as outras funcionalidades do sistema se apoiam.

\section{Ponto de vista do desenvolvimento}

O desenvolvimento será feito em Python. Com base nas bibliotecas
Django~\cite{django} para as funcionalidades baseadas em web e
Numpy~\cite{numpy} para cálculos numéricos. O estilo de código deve seguir o
definido pelo PEP 8~\cite{pep8}.

São obrigatórios testes funcionais a nível de serviço, ficando à discrição do
desenvolvedor os locais mais oportunos para utilizar também testes unitários.

\subsection{Arquitetura hexagonal} 

A estrutura interna de cada módulo segue o padrão comumente denominado de
\textit{arquitetura
hexagonal}~\cite{cockburn_hexagonal}~\cite{fowler2015layering}, uma derivação
da arquitetura clássica em três camadas que utiliza \textit{Data
Mappers}~\cite{fowler2002patterns} para eliminar dos objetos de domínio a
dependência da camada de acesso a dados. A estrutura geral deste padrão
arquitetural pode ser observada na Figura~\ref{hexagonal}.

\begin{figure}[h]
	\centering
	\includegraphics[]{hexagonal}
        \caption{Diagrama esquemático da arquitetura hexagonal. Adaptado de \citeonline{fowler2015layering}.\label{hexagonal}}
\end{figure}


\section{Ponto de vista da implantação}

Para atender os requisitos de segurança e disponibilidade exigidos nos módulos
computacionais, será utilizada a infraestrutura apresentada na
Figura~\ref{implantacao}.

\begin{figure}[h]
	\centering
	\includegraphics[]{implantacao}
	\caption{Diagrama de infraestrutura\label{implantacao}}
\end{figure}

Neste diagrama se observa a presença de um servidor interno à fábrica,
necessário para garantir confiabilidade aos módulos Alocação e Comunicação, que
caso contrário seriam vulneráveis a falhas na conexão da fábrica à internet.  O
servidor externo agrega informações de performance das operações de diversas
fábricas de forma a gerar relatórios executivos para utilização na camada
estratégica.

Decidiu-se utilizar um único servidor interno à fábrica porque, ainda que
houvesse redundância no servidor de aplicação, o servidor banco de dados ainda
seria um ponto único de falha. A duplicação em redundância ativa de toda a
infraestrutura foi considerada muito custosa, mas poderá ser realizada caso
haja demanda.

Os módulos da aplicação serão implantados na forma de contêineres Docker,
operando atrás de um servidor \textit{web} Nginx para tratar potenciais
clientes lentos e servir arquivos estáticos.

\chapter{Estudo de caso}\label{resultados}

Obteve-se os dados operacionais dos processos de manutenção preventiva de uma
planta industrial no estado de São Paulo, cujo modelo segue o da
Figura~\ref{preventiva}. Os dados referem-se aos meses de junho, julho e agosto
de 2017, contendo informações a respeito de:

\begin{enumerate}

    \item Recursos que executaram as tarefas;

    \item Máquinas que foram objeto da manutenção;

    \item Natureza do trabalho (por exemplo: manutenção mecânica ou elétrica)

    \item Data e hora de início e fim da instância do processo.

    \item Esforço despendido nas tarefas.

\end{enumerate}

Este banco de dados não está no formato tradicional utilizado em mineração de
processos de negócio, por estar orientado não a eventos, mas a instâncias. Para
simplificar a análise dos resultados, escolheu-se um tipo de processo para
avaliar: a manutenção mecânica de fornos elétricos de uma das áreas da fábrica.
Um exemplo ilustrativo da estrutura do banco de dados estrutura encontra-se na
Figura~\ref{exemplobanco}.

\begin{figure}[h]
	\centering
	\includegraphics[width=\textwidth]{exemplo}
	\caption{Exemplo ilustrativo da estrutura do banco de dados após limpeza.\label{exemplobanco}}
\end{figure}

A coluna `manutencionista' contém um código de identificação dado a cada
recurso humano que atua na manutenção. A coluna `trabalho' é o esforço
despendido na tarefa de manutenção, medido em minutos. As colunas de horário de
início e fim se referem à duração total da instância, incluindo as tarefas de
aquisição de ferramentas necessárias e locomoção até o local.

\begin{figure}[h]
	\centering
	\includegraphics[width=\textwidth]{trabreal}
        \caption{Esforço médio despendido na tarefa de manutenção para cada manutencionista que a executou no período analisado.\label{trabreal}}
\end{figure}

Para avaliar o impacto que a aplicação do sistema de planejamento da alocação
de recursos executores teria neste processo, é preciso validar a hipótese de
que a escolha de recurso utilizado para executar as tarefas influencia na
performance do processo. Neste caso, a métrica utilizada será a quantidade de
trabalho despendido. A Figura~\ref{trabreal} mostra o esforço médio e seu
desvio padrão para cada manutencionista analisado, para a tarefa escolhida.
Como se pode verificar, não apenas a escolha de recurso implica, na média, em
uma diferença no esforço despendido, como também na consistência dos resultados
obtidos.

A seguir, criou-se o modelo do processo operacional. Dadas as restrições nos
dados disponíveis, o processo original foi simplificado em duas atividades,
como demonstrado na Figura~\ref{simplificado}. A tarefa de preparação foi
inclusa para poder considerar nas métricas a diferença entre a duração total da
instância do processo e o esforço despendido na tarefa de manutenção de fato.
Assim, a métrica definida associada a este processo foi a soma das durações das
duas tarefas, equivalente à duração total da instância.

\begin{figure}[h]
	\centering
	\includegraphics[width=\textwidth]{simplificado}
        \caption{Modelo simplificado do processo de manutenção preventiva para uso no sistema.\label{simplificado}}
\end{figure}

Treinando o modelo preditivo para a métrica de performance deste processo, com
uso de validação cruzada em \textit{3-fold}, obteve-se um coeficiente de
determinação $r^2~=~0.83$, considerável aceitável para o uso em planejamento.

\chapter{Conclusões}

O sistema proposto obteve resultados positivos na predição da performance
operacional do processo de manutenção preventiva mecânica dos fornos elétricos
na planta industrial estudada. O processo estudado, por ser notoriamente
simples, atendeu todas as hipóteses feitas a respeito de sua estrutura, e o
estudo dos dados validou para este processo a hipótese implícita de influência
das decisões de alocação de recursos executores no resultado operacional. O
modelo preditivo proposto obteve um resultado aceitável para, a elaboração de
planos de alocação. Porém, como não havia dados a respeito do planejamento das
atividades, não foi possível testar a qualidade de planos gerados com base
neste modelo.

Os resultados obtidos revelam oportunidades de estudos futuros. A citar:

\begin{itemize}

    \item Análise das características dos processos onde sistemas como este
            apresentam maior valor.

    \item Melhorias no algoritmo de planejamento para relaxar hipóteses
            estruturais feitas a respeito do processo operacional.

    \item Análise da performance do modelo preditivo de desempenho em processos
            mais complexos. Há evidências, como em
                \citeonline{folino2012discovering}, de que pode ser necessário
                um modelo mais complexo.

    \item Adoção de um modelo sensível a risco para o planejamento, como o
            proposto em \citeonline{freire2016extreme}

\end{itemize}


% ========== Referências ==========
\bibliographystyle{abntex2-alf}
\bibliography{refs}

\end{document}
